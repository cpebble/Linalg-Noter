\documentclass[a4paper,fleqn]{article}
\title{Rapport}
\author{P\aa b\o l}

\usepackage{amssymb}
\usepackage{fancyhdr}
\usepackage{amsmath}
\usepackage{tikz}
\usepackage{pgfplots}
\usepackage{graphicx}
\graphicspath{ {images/} }
\usepackage[danish]{babel}
\usepackage[utf8]{inputenc}
\usepackage{lastpage}
\usepackage{lipsum}
\usepackage[colorlinks, linkcolor=black]{hyperref}
\usepackage{listings}
\usepackage{upquote}

\newcommand{\hmwkTitle}{Projekt 1} % Assignment title
\newcommand{\hmwkDueDate}{03/05\ -\ 2019} % Due date
\newcommand{\hmwkClass}{Lineær Algebra} % Course/class
\newcommand{\hmwkClassInstructor}{Lærer: Henrik$^2$} % Teacher/lecturer
\newcommand{\hmwkAuthorName}{Christian P\aa b\o l: \emph{wbr220}} % Your name
\newcommand{\hmwkProblem}{Problemformulering: \emph{Ligningssystemer}}

\addtolength{\oddsidemargin}{-.875in}
\addtolength{\evensidemargin}{-.875in}
\addtolength{\textwidth}{1.75in}
\setlength{\parindent}{0in}

\pagestyle{fancy}
\lhead{\hmwkAuthorName} % Top left header
\chead{\hmwkClass\ : \hmwkTitle} % Top center head
\rhead{\rightmark}
\cfoot{} % Bottom center footer
\rfoot{Page\ \thepage\ of\ \protect\pageref{LastPage}} % Bottom right footer

\title{
	\vspace{2in}
	\textmd{\textbf{\hmwkClass:\ \hmwkTitle}}\\
	\normalsize\vspace{0.1in}\small{Afleveres:\ \hmwkDueDate}\\
	\vspace{0.1in}\large{\textit{\hmwkClassInstructor}}\\
	\normalsize\vspace{0.5in} \hmwkProblem  \\
	\vspace{3in}
}

\author{\textbf{\hmwkAuthorName}}

\begin{document}
	\maketitle
	\newpage
	\setcounter{page}{1}
	\section{Opgave 1}
	Vi betragter ligningssystemet med det ukendte, reele tal $a$:
	\[ (S) = \left\{\begin{matrix}
			2x_1 + (3+a)x_2 + 2x_3 = 2+a\\
			x_1 + ax_2 + 2x_3 = a\\
			ax_1 + 2x_2 + 2ax_3 = 0\\
	\end{matrix}\right.\]
	\subsection{(a)}
	Antag at $A \neq \pm \sqrt{2}$. Vi bliver så bedt om at løse ligningssystemet ved at sætte
	totalmatricen på reduceret rækkeechelonform\\
	Vi har totalmatricen \[A = \begin{bmatrix}2&3+a&2\\1&a&2\\a&2&2a\end{bmatrix} \quad 
	b = \begin{bmatrix}2+a \\ a \\ 0\end{bmatrix} \]
	Matricen vi gerne vil eliminere er derfor:
	\[ [A|b] = \left[\begin{array}{@{}ccc|c@{}}2&3+a&2 & 2+a\\1&a&2 & a\\a&2&2a & 0\end{array} \right]\]
	Vi bruger Gauss-Jordan elimination til at løse ligningssystemet\\
	\[ [A|b] = \left[\begin{array}{@{}ccc|c@{}}
		2	&	3+a	&	2	& 2+a\\
		1	&	a	&	2	& a\\
		a	&	2	&	2a	& 0
	\end{array} \right] r_1 \leftrightarrow r_2 \]
	\[ \left[\begin{array}{@{}ccc|c@{}}
		1	&	a	&	2	& a\\
		2	&	3+a	&	2	& 2+a\\
		a	&	2	&	2a	& 0
	\end{array} \right] r_2 - 2r_1 \rightarrow r_2, r_3 -ar_1 \rightarrow r_3\]
	\[ \left[\begin{array}{@{}ccc|c@{}}
		1	&	a	&	2	& a\\
		0	&	3 - a	&	-2	& 2-1a\\
		0	&	-a^2+2	&	0	& -a^2
	\end{array} \right] \frac{1}{3-a}r_2 \rightarrow r_2 \]
	\[ \left[\begin{array}{@{}ccc|c@{}}
		1	&	a		&	2		& a\\
		0	&	1		&	\frac{2}{-3+a}	& \frac{-2+a}{-3+a}\\
		0	&	-a^2+2		&	0		& -a^2
	\end{array} \right] r_1 - ar_2 \rightarrow r_1 \]
	\emph{Her bryder jeg algoritmen lidt, og begynder at lave backwards reduction. Det kommer
	til at hjælpe mig om et par trin}
	\[ \left[\begin{array}{@{}ccc|c@{}}
		1	&	0		&	-\frac{6}{-3+a}	& -\frac{a}{-3+a}\\
		0	&	1		&	\frac{2}{-3+a}	& \frac{-2+a}{-3+a}\\
		0	&	-a^2+2		&	0		& -a^2
	\end{array} \right] r_3 - (-a^2+2)r_2 \rightarrow r_3 \]
	\[ \left[\begin{array}{@{}ccc|c@{}}
		1	&	0		&	-\frac{6}{-3+a}	& -\frac{a}{-3+a}\\
		0	&	1		&	\frac{2}{-3+a}	& \frac{-2+a}{-3+a}\\
		0	&	0		&	2\frac{a^2-2}{-3+a}		& \frac{a^2-2a+4}{-3+a}
\end{array} \right] \frac{1}{2 - a^2} r_3 \rightarrow r_3 \]
\emph{Nu får vi glæde af den tidligere ERO. Alt i kolonne 3 går netop op i $A'_{33}$ og kolonnen løses med substitution}
	\[ \left[\begin{array}{@{}ccc|c@{}}
		1	&	0		&	-\frac{6}{-3+a}	& -\frac{a}{-3+a}\\
		0	&	1		&	\frac{2}{-3+a}	& \frac{-2+a}{-3+a}\\
		0	&	0		&	-\frac{2}{-3+a}	& -\frac{a^2-2a+4}{(-3+a)(a^2-2)}
\end{array} \right] r_1 - r_3 \rightarrow r_1, \quad r_2 + r_3 \rightarrow r_2 \]
	\[ \left[\begin{array}{@{}ccc|c@{}}
		1	&	0		&	0		& -\frac{a^2+4}{a^2 -2}\\
		0	&	1		&	0		& \frac{a^2}{a^2-2}\\
		0	&	0		&	-\frac{2}{-3+a}	& -\frac{a^2-2a+4}{(-3+a)\cdot(a^2-2)}
\end{array} \right] (-(1/2)*(-3+a))r_3 \rightarrow r_3\]
	\[ \left[\begin{array}{@{}ccc|c@{}}
		1	&	0		&	0		& -\frac{a^2+4}{a^2 -2}\\
		0	&	1		&	0		& \frac{a^2}{a^2-2}\\
		0	&	0		&	1		& \frac{1}{2}\cdot\frac{a^2-2a+4}{a^2-2}
\end{array} \right] \]
Vi har nu det løste ligningssystem hvor
\[ x = \begin{bmatrix}
		-\frac{a^2+4}{a^2 -2}\\
		\frac{a^2}{a^2-2}\\
		\frac{1}{2}\cdot\frac{a^2-2a+4}{a^2-2}
\end{bmatrix}\]

	\subsection{(b)}
	Antag nu at $a = \pm\sqrt{2}$. Løs den nye matrice på metoden som før\\
	Vi kigger på matricen fra forrige opgave:
	\[ [A|b] = \left[\begin{array}{@{}ccc|c@{}}
		2	&	3+a	&	2	& 2+a\\
		1	&	a	&	2	& a\\
		a	&	2	&	2a	& 0
	\end{array} \right]\]
	Vi konkluderer så med rækkeoperationen $r_3 - \sqrt(2)r_2 \rightarrow r_3$ at systemet er
	inconsistent da $Rank A < Rank [A|b]$:
	\[ [A|b] = \left[\begin{array}{@{}ccc|c@{}}
		2	&	3+a	&	2	& 2+a\\
		1	&	a	&	2	& a\\
		0	&	0	&	0	& -2
	\end{array} \right]\]

	\subsection{(c)}
	Vi så i (A) at ligningssystemet på rækkeechelonform har Rank A = n hvis $a \neq\pm\sqrt{2}$.
	I bogens Theorem 1.3 står der \emph{"If rank A = n then (S) has a unique solution"}.
	Derfor er der en og kun en løsning.

	\subsection{(d)}
	Jeg antager nu $a = -2$ og vil bestemme den inverse matrix til koefficientmatricen $A$ ved
	hjælp af \verb!COMPUTATION! algoritmen til at udregne $A^{-1}$. Jeg vil dernæst udregne
	$A^{-1}\cdot\begin{bmatrix}0\\-2\\0\end{bmatrix}$

	Computation siger at for en given $n x n$ matrice at $[A | I_n] \tilde [I_n | A^{-1}]$\\
	Vi opstiller
	\[ [A|I_n] = \left[\begin{array}{@{}ccc|ccc@{}}
			2	&	1	&	2	&	1	&	0	&	0\\
			1	&	-2	&	2	&	0	&	1	&	0\\
			-2	&	2	&	-4	&	0	&	0	&	1\\
	\end{array}\right] r_1 \leftrightarrow r_2, r_2 - 2r_1 \rightarrow r_2, r_3 + 2r_1 \rightarrow r_3\]
	\[  \left[\begin{array}{@{}ccc|ccc@{}}
			1	&	-2	&	2	&	0	&	1	&	0\\
			0	&	5	&	-2	&	1	&	-2	&	0\\
			0	&	-2	&	0	&	0	&	2	&	1\\
	\end{array}\right] r_2 \leftrightarrow r_3 \]
	\[  \left[\begin{array}{@{}ccc|ccc@{}}
			1	&	-2	&	2	&	0	&	1	&	0\\
			0	&	-2	&	0	&	0	&	2	&	1\\
			0	&	5	&	-2	&	1	&	-2	&	0\\
	\end{array}\right]  2r_3 \rightarrow r_3 \]
	\[  \left[\begin{array}{@{}ccc|ccc@{}}
			1	&	-2	&	2	&	0	&	1	&	0\\
			0	&	-2	&	0	&	0	&	2	&	1\\
			0	&	10	&	-4	&	2	&	-4	&	0\\
	\end{array}\right] r_3 + 5r_2 \rightarrow r_3, r_1 - r_2 \rightarrow r_1\]
	\[  \left[\begin{array}{@{}ccc|ccc@{}}
			1	&	0	&	2	&	0	&	-1	&	-1\\
			0	&	-2	&	0	&	0	&	2	&	1\\
			0	&	0	&	-4	&	2	&	6	&	5\\
	\end{array}\right] -\frac{1}{2}r_2 \rightarrow r_2, -\frac{1}{4}r_3 \rightarrow r_3\]
	\[  \left[\begin{array}{@{}ccc|ccc@{}}
			1	&	0	&	2	&	0	&	-1	&	-1\\
			0	&	1	&	0	&	0	&	-1	&	-\frac{1}{2}\\
			0	&	0	&	1	&	-\frac{1}{2}&	-\frac{3}{6}	& -\frac{5}{4}\\
	\end{array}\right] r_1 - 2r_2 \rightarrow r_1 \]
	\[  \left[\begin{array}{@{}ccc|ccc@{}}
			1	&	0	&	0	&	1	&	2	&	\frac{3}{2}\\
			0	&	1	&	0	&	0	&	-1	&	-\frac{1}{2}\\
			0	&	0	&	1	&	-\frac{1}{2}&	-\frac{3}{2}	& -\frac{5}{4}\\
	\end{array}\right]\]

	
	Vi har derfor \[A' = \begin{bmatrix}
		1	&	2	&	\frac{3}{2}\\
		0	&	-1	&	-\frac{1}{2}\\
		-\frac{1}{2}&	-\frac{3}{2}	& -\frac{5}{4}\\
	\end{bmatrix}\]

	\[ 0 + 2\cdot -2 + 0 = -4 ; 0 + 4 + 0\]
	\[A'\begin{bmatrix}0\\-2\\0\end{bmatrix} = \begin{bmatrix}-4\\2\\3\end{bmatrix}\]

	\section{Opgave 2}
	Vi har en $3 x 3$ matrice $A$ som ved rækkeoperationer $ero_1, ero_2, ero_3, ero_4$ kan
	omdannes til enhedsmatricen. Vi har
	\[ ero = \begin{bmatrix}
			r_2 -4r_1 \rightarrow r_2\\
			-\frac{1}{5}r_3 \rightarrow r_3\\
			r_1 \leftrightarrow r_3\\
			r_1 + 10r_3  \rightarrow r_1
	\end{bmatrix}\]
	\subsection{(a)}
	Vi skal for hver af de fire rækkeoperationer udregne den elementære matrice $E_{ij}$ som
	svarer til rækkeoperationen. Du finder elementærmatricen ved at udføre rækkeoperationen på
	Identitetsmatricen $I_3$\\
	Vi Udregner $E_n$\\
	\[ E_1 = \begin{bmatrix}1&0&0\\0&1&0\\0&0&1\end{bmatrix} r_2 - 4r_1 \rightarrow r_2 
	\begin{bmatrix} 1 & 0 & 0 \\ -4 & 1 & 0 \\ 0 & 0 & 0 \end{bmatrix} \]
	\[ E_2 = \begin{bmatrix}1&0&0\\0&1&0\\0&0&1\end{bmatrix} -\frac{1}{5}r_3 \rightarrow r_3
	\begin{bmatrix}1&0&0\\0&1&0\\0&0&-\frac{1}{5}\end{bmatrix}\]
	\[ E_3 = \begin{bmatrix}1&0&0\\0&1&0\\0&0&1\end{bmatrix} r_1 \leftrightarrow r_3 
	\begin{bmatrix}0&0&1\\0&1&0\\1&0&0\end{bmatrix}\]
	\[ E_4 = \begin{bmatrix}1&0&0\\0&1&0\\0&0&1\end{bmatrix} r_1 + 10r_3 
	\begin{bmatrix}1&0&10\\0&1&0\\0&0&1\end{bmatrix}\]

	Der står i lærebogen: "Performing an elementary row operation... accomplished by
	premultiplying $A$ by an elementary matrix". Altså kan vi sige at hvis vi udfører $ero_1$
	på matrice $A$ får vi $A' = E_1A$.\\

	Vi ved at vi kommer frem til identitestmatricen $I_3$ hvis vi udfører de fire 
	rækkeoperationer $ero_n$ på $A$. Vi opstiller derfor $I_3 = A^{''''}$. Vi opstiller nu:
	\begin{equation}
	\begin{array}{l @{=} l}
		A^{'} & E_1A\\
		A^{''} & E_2(E_1A)\\
		A^{'''} & E_3(E_2(E_1A))\\
		A^{''''}& E_4(E_3(E_2(E_1A)))\\
		I_3 & E_4E_3E_2E_1A
	\end{array}
	\end{equation}

	\subsection{(b)}
	Vi udregner nu for hvert af de fire elemmentærmatricer den inverse matrice $E_n^{-1}$\\
	For at få den inverse matrix for en elementærmatrice kan vi bruge den inverse
	rækkeoperation(se Theorem 1.1)\\
	Vi opstiller(Hvor $F_n$ er $E_n^{-1}$)
	\[ F_1 = \begin{bmatrix}1&0&0\\0&1&0\\0&0&1\end{bmatrix} r_2 + 4r_1 \rightarrow r_2 
	\begin{bmatrix} 1 & 0 & 0 \\ 4 & 1 & 0 \\ 0 & 0 & 1 \end{bmatrix} \]
	\[ F_2 = \begin{bmatrix}1&0&0\\0&1&0\\0&0&1\end{bmatrix} -5r_3 \rightarrow r_3
	\begin{bmatrix}1&0&0\\0&1&0\\0&0&-5\end{bmatrix}\]
	\[ F_3 = \begin{bmatrix}1&0&0\\0&1&0\\0&0&1\end{bmatrix} r_1 \leftrightarrow r_3 
	\begin{bmatrix}0&0&1\\0&1&0\\1&0&0\end{bmatrix}\]
	\[ F_4 = \begin{bmatrix}1&0&0\\0&1&0\\0&0&1\end{bmatrix} r_1 - 10r_3 
	\begin{bmatrix}1&0&-10\\0&1&0\\0&0&1\end{bmatrix}\]

	Vi vil nu gerne bestemme $A$. Det gør vi ved at udføre de inverse matrix operationer på $A$
	i omvendt rækkefølge. Vi skal derfor udregne $F_1 \cdots F_4 I_3$\\
	Vi b udregner\footnote{Udregninger udført i Maple\texttrademark}
	\[ F_1 \cdot F_2 \cdot F_3 \cdot F_4 \cdot I_3 = A = \begin{bmatrix}
		0&0&1\\0&1&4\\-5&0&50
	\end{bmatrix}\]
	For at checke efter udfører vi nu udregningen fra forrige opgave $I_3 = E_4E_3E_2E_1A$
	\[
	i_3 = E_4\cdot E_3\cdot E_2\cdot E_1\cdot A = \begin{bmatrix}1&0&0\\0&1&0\\0&0&1\end{bmatrix}
	\]
	
	\subsection{(c)}
	Vi får nu givet \[B = \begin{bmatrix}0&1\\0&4\\-5&50\end{bmatrix}\] og
	\[X = \begin{bmatrix}10&0&-\frac{1}{5}\\1&0&0\end{bmatrix}\]
	Vi skal nu vise at $X$ er venstre invers til $B$:
	\[XB = \begin{bmatrix}10&0&-\frac{1}{5}\\1&0&0\end{bmatrix} \begin{bmatrix}0&1\\0&4\\-5&50\end{bmatrix}
		= \begin{bmatrix}
			-5\cdot -\frac{1}{5} & (1\cdot 10) + (-\frac{1}{5}\cdot 50) \\
			0 & 1\cdot 1
		\end{bmatrix} = \begin{bmatrix}1&0\\0&1\end{bmatrix}
	\]
	Derudover skal vi finde andre venstreinverse matricer til $B$. Vi kan se på det som en
	ligning $XB = I_{2}$. Vi kan i dette tilfælde se at identitetsmatricen er $I_2$ da
	multiplikation af en $n x m$ matrice og en $m x n$ matrice giver en $n x n$ matrice. Vi
	opstiller:
	\[\begin{bmatrix}x_1 & x_2 & x_3 \\ x_4 & x_5 & x_6\end{bmatrix}
	\begin{bmatrix}0&1\\0&4\\-5&50\end{bmatrix} =
	\begin{bmatrix}
		-5x_3 	& x_1+4x_2+50x_3	\\
		-5x_6 	& x_4 + 4x_5 + 50x_6 \\
	\end{bmatrix} = 
	\begin{bmatrix}1&0\\0&1\end{bmatrix} \]
	Vi ser på det som et ligningssystem, og opskriver
	\[ (S) = \left\{ \begin{array}{cccccc @{= } l}
		&&-5x_3 & & & & 1\\
		x_1& +4x_2&+50x_3 &&&& 0\\
		   &&&&&-5x_6&0\\
		   &&&x_4&+4x_5&+50x_6&1
	\end{array}\right. \]
	Vi sætter så systemet på rækkeechelonform
	\[\begin{bmatrix}
		0&0&-5&0&0&0&1\\
		1&4&50&0&0&0&0\\
		0&0&0&0&0&-5&0\\
		0&0&0&1&4&50&1\\
\end{bmatrix} r_1 \leftrightarrow r_2, r_3 \leftrightarrow r_2, -\frac{1}{5}r_2 \rightarrow r_2, -\frac{1}{5}r_4 \rightarrow r_4\]
	\[\begin{bmatrix}
		1&4&50&0&0&0&0\\
		0&0&1&0&0&0&-\frac{1}{5}\\
		0&0&0&1&4&50&1\\
		0&0&0&0&0&1&0\\
	\end{bmatrix} r_1 - 50 r_2 \rightarrow r_1, r_3 - 50r_4 \rightarrow r_3\]
	\[\begin{bmatrix}
		1&4&0&0&0&0&10\\
		0&0&1&0&0&0&-\frac{1}{5}\\
		0&0&0&1&4&0&1\\
		0&0&0&0&0&1&0\\
	\end{bmatrix} \]
	Og vi har nu systemet
	\[
		\begin{bmatrix}x_1\\x_2\\x_3\\x_4\\x_5\\x_6\end{bmatrix} = 
		\begin{bmatrix}10\\0\\-\frac{1}{5}\\1\\0\\0\end{bmatrix} + s
		\begin{bmatrix}-4\\1\\0\\0\\0\\0\end{bmatrix} + t
		\begin{bmatrix}0\\0\\0\\-4\\1\\0\end{bmatrix}
	\]
	Vi skriver nu $X$ op:
	\[
	\begin{bmatrix}
	10-4s & s & -\frac{1}{5}\\ 1 -4t & t & 0
	\end{bmatrix}
	\]
	For at checke efter bruger vi den udleverede matrice $X$ fra tidligere. Der sætter vi 
	$s = t = 0$ og får:
	\[
		\begin{bmatrix}
			10 & 0 & -\frac{1}{5}\\
			1 & 0 & 0
		\end{bmatrix}
	\] Vi ser hermed at Løsningen er valid\\


	Matrice $B$ har derudover ingen højreinvers. Bogens Theorem 2.10 siger at $B$ har en
	højreinvers hvis
	og kun hvis matricens rank er lig antal rækker, og der er færre end eller lige så mange 
	rækker som kolonner. Man kan tydeligt se at matricen er rang 2, og at den har flere rækker
	end kolonner.\\

	\section{C}
	Jeg har fået given en DiGraph $G$ med fem knuder. 

	\subsection{(a)}
	Jeg er blevet bedt om at ræpresentere grafen som en nabomatrice.\\
	\begin{equation}
		N = \begin{bmatrix}
			0 & 1 & 1 & 1 & 0\\
			0 & 0 & 0 & 1 & 1\\
			0 & 0 & 0 & 1 & 0\\
			1 & 0 & 0 & 0 & 1\\
			1 & 0 & 1 & 0 & 0\\
		\end{bmatrix}
	\end{equation}

	Det smarte ved en nabomatrice er at vi kan se hvilke knuder der er forbundet, og dermed
	hvilke veje af længde $k = 1$ der findes fra knude $i \rightarrow j$. Er vi interreseret i
	at finde hvor mange veje fra $i \rightarrow j$ med en arbitrær længde $k \in \mathbb{Z}^+$,
	kan vi aflæse det af $N^k$, hvor hvert komponent $N^k_{ij} = \text{Antal veje fra i til j}$
	
	Vil vi læse antal veje fra knude $i = 2$ til knude $j = 5$ af længde $k = 8$ skal vi derfor
	bruge $N^8$
	\[
	N^8 = \begin{bmatrix}
		228&100&227&144&145\\
		158&70&158&100&100\\
		179&79&180&114&113\\
		179&79&179&114&114\\
		179&79&179&114&114\\
	\end{bmatrix}\]
	Hvor vi kan aflæse $N^8_{25} = 100$

	\subsection{(b)}
	Denne graf kan bruges til at ræpresentere et web med fem sider. Vi vil nu opstille
	linkmatricen for denne graf. Det skrives at linkmatricen $A$ har elementer $1/N_j$ hvis der
	fremkommer en kant fra $i$ til $j$. Vi går igennem matricen og erstatter alle elementer
	$a = 1$ med $a = \frac{1}{N_j}$ \\
	\begin{equation}
		\begin{bmatrix}
			0&1&\frac{1}{2}&\frac{1}{3}&0\\
			0&0&0&\frac{1}{3}&\frac{1}{2}\\
			0&0&0&\frac{1}{3}&0\\
			\frac{1}{2}&0&1&0&\frac{1}{2}\\
			\frac{1}{2}&0&\frac{1}{2}&0&0\\
		\end{bmatrix}
	\end{equation}

	\subsection{(c)}
	Til sidst skal vi nu finde en vektor $x \neq 0$ som opfylder ligningen $Ax = x$ og
	rangordne siderne. $Ax = x$ er noget af en "sjov" ligning, så vi trækker $x$ fra og
	opstiller $Ax - x = 0$. Dette er som beskrevet i dokumentet Google's page rank.\\
\[[Ax - x|0_5]\left[\begin{array}{ccccc|c}
	-1		&1	&\frac{1}{3}	&\frac{1}{2}	&0 		&0\\
	0		&-1 	&0		&\frac{1}{2}	&\frac{1}{2}	&0\\
	\frac{1}{3}	&0	&-1		&0		&\frac{1}{2}	&0\\
	\frac{1}{3}	&0	&\frac{1}{3}	&-1		&0		&0\\
	\frac{1}{3}	&0	&\frac{1}{3}	&0		&-1		&0\\
\end{array}\right]6r_1, 2r_2, 6r_3, 3r_4, 3r_5\]

$ero_1: r_4 \leftrightarrow r_1$\input{echelons/ero1.tex}
$ero_2: r_3 - 2r_1 \rightarrow r_3, r_4 + 6r_1 \rightarrow r_4, r_5 -1r_1\rightarrow r_5$\input{echelons/ero2.tex}
$ ero_3: r_4 + 3r_2 \rightarrow r_4$\input{echelons/ero3.tex}
$ero_4: r_4 + r_3 \rightarrow r_4$ \input{echelons/ero4.tex}
$ ero_5: \frac{1}{6}r_4 \rightarrow r_4, r_5 + 3r_4 $\input{echelons/ero5.tex}
	Vi er nu på rækkeechelonform\\\input{echelons/rre.tex}

	Vi opstiller følgende transformationer:\\
$r_3 + \frac{3}{4}r_4 \rightarrow r_3, r_2 + \frac{1}{2}r_4 \rightarrow r_2, r_1 + 3r_4 \rightarrow r_1$\input{echelons/rre1.tex}
$r_1 -1r_3 \rightarrow r_1$\input{echelons/rre2.tex}

	Vi skriver nu løsningssystemet op:
\[
	\begin{bmatrix}x_1\\x_2\\x_3\\x_4\\x_5\end{bmatrix} =
	t\cdot \begin{bmatrix}\frac{15}{8}\\1\\\frac{9}{8}\\1\\1\end{bmatrix}
\]
	Vi kan derved aflæse skoren af de forskellige sider, og sætter dem i følgende rækkefølge:
	$x_1 > x_3 > x_2 = x_4 = x_5$ altså, den side med højest skore er $x_1$

\end{document}
