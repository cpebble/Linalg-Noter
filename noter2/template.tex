\documentclass[a4paper,fleqn]{report}
\title{Rapport}
\author{P\aa b\o l}

\usepackage{fancyhdr}
\usepackage{amsmath}
\usepackage{amssymb}
\usepackage{amsthm}
\usepackage{tikz}
\usepackage{pgf}
\usepackage{pgfplots}
\usepackage{graphicx}
\graphicspath{ {images/} }
\usepackage[danish]{babel}
\usepackage[utf8]{inputenc}
\usepackage{lastpage}
\usepackage{lipsum}
\usepackage[colorlinks, linkcolor=black]{hyperref}
\usepackage{listings}
\usepackage{upquote}

\setlength{\parindent}{0pt}

\newcommand{\RR}{\mathbb{R}}
\newtheorem{example}{Eksempel}[chapter]
\newtheorem{theorem}{Theorem}[chapter]

\begin{document}
	\maketitle
	\setcounter{page}{1}
	\chapter{Uge 2: 30/04}
	\section{Kvadratiske matricer}
	En $NxN$ matrix $D$ er en diagonalmatrix hvis $d_ij = 0$ for $i \neq j$\\

	Enhedsmatricen $I_n$ er denne $nxn$ diagonalmatrix
	\[\begin{bmatrix}1&0&\dots&0\\0&1&\dots&\dots\\0&\dots&0&1\end{bmatrix}\]

	\section{Inversible matricer}
	En $n x n$ matrice er invertibel hvis der findes en $n x n$ matrix så \[XA = AX = I_n\]
	Altså hvis både AX og XA giver identitetsmatricen\\
	Hvis både $X_1$ og $X_2$ er inverse til $A$ så gælder: 
	\[X_2 = X_2I_n = X_2(AX_1) = (X_2A)X_1 = I_nX_1 = X_1\]
	Derfor er den inverse matrice entydigt bestemt og betegnes $A_{-1}$\\
	En invertibel matrix $A$ kaldes også regulær, ellers kaldes den singulær\\
	Hvis $A$ er invertibel så har ligningen $Ax = y$ netop en løsning $x = A^{-1}y$\\

	\subsection{Theorem p. 85}
	Hvis $A$ og $B$ er invertible så er $AB$ også invertibel og der gælder $(AB)^{-1} = B^{-1}
	A^{-1}$\\
	Hvis A er invertibel så er $A^T$ også invertibel og $(A^T)^{-1} = A$\\
	Hvis A er invertibel så er $A^{-1}$ også invertibel og $(A^{-1})^{-1} = A$\\
	Hvis $A$ er invertibel og $S \neq 0$ så er $sA$ også invertibel og $s^{-1}A^{-1}=(sA)^{-1}$

	\subsection{Def 2.8}
	En $N x N$ matrix $X$ er højreinvers til $A$ hvis $AX = I_n$\\
	Hvis $XA = I_n$ er X venstreinvers\\

	\subsection{Theorem 2.4}
	Hvis der findes netop en matrice X som er en højreinvers til A så er X også venstreinvers
	til A og dermed $X = A^{-1}$

	\section{Bestemmelse af invers ved rækkeoperationer}
	Lad $A$ være en $N x N$ matrice og lad $I = I_n$\\
	\begin{enumerate}
		\item Opskriv nmatricen $[A|I]$
		\item Bring matricen $[A|I]$ til reduceret rækkeechelonform
		\item Hvis den reducerede rækkeechelonform er $[I|X]$ da er $A^-1 = X$ hvis ikke 
			da er A ikke invertibel
	\end{enumerate}
	
	\chapter{07/05}
	\section{Vekrorrummet Rn}
	Der gælder pene rægneregler\\
	Multiplikation er ikke heeeeelt simpelt\\

	En mægde V hvis elementer er vektorer med\\
	\begin{itemize}
		\item Vektoraddition, $V x V \rightarrow V, (u,v) \rightarrow u + v$
		\item Skalarmultiplikation $R  V \rightarrow V, (s,u) \rightarrow su$
	\end{itemize}
	Mængden $V = R^n$ er et reelt vektorrum\\

	\textbf{VI holder os fra abstrakte vektorrum} Men alt hvad vi skal lære om $R^n$ gælder
	også i et abstrakt vektorrum $V$

	\section{Underrum}
	Pæne delmængder i $\RR^n$\\
	En delmængde $u \subseteq \RR^N $ har tre krav\\
	Nulvektorern skal ligge i $u$\\
	For alle $u,v \in u$ gælder også $u+v \in u$\\
	For alle $s \in \RR$ og $u \in U$ gælder det $su \in U$\\
	Eksempler er:
	$U = \{(0,0)\}, U = \text{en linie gennem 0,0}, U = \RR$ \\

	Hvad kan man bruge underrum til? Principal component analysis. Der tilnærmer man et 
	højdimensionalt datasæt med et lavdimensionalt underrum.\\

	\begin{example}
		Vi går til $\RR^4$.\\
		$u = \{(x_1,x_2,x_3,x_4)\} \in \RR^4 | x_1 + x_2 = 2$ og $x_2 + x_3 + x_4 = 0$\\
		Vi opstiller følgende operation\\
		\[ x = \begin{pmatrix}x_1\\x_2\\x_3\\x_4\end{pmatrix} \Leftrightarrow
			\begin{pmatrix} 1&1&0&0\\0&1&1&1\end{pmatrix}\begin{pmatrix}x_1\\x_2\\x_3\\x_4\end{pmatrix}
		= \begin{pmatrix}0\\0\end{pmatrix} \Leftrightarrow Ax = 0\]
		Da $A0 = 0$ gælder $0 \subseteq U$\\
		Hvis u,v $\subset u$ gælder $Au = 0$ og $Av = 0$. Matrixregning giver
		\[ A(u+v) = Au + Av = 0 + 0 = 0\]
		og dermed gælder $u+v \subset U$\\
		Lad $s \in \RR$. Det gælder at $s(u+v) \subset U$ og da 
		\[ A(su) = sAu = s\cdot0 = 0\]
	\end{example}


	\section{Span}
	Lad $S = \{ v_1, \dots, v_k\} \in \RR$. Sæt \\
	$span S = span\{v_1,\dots,v_k\} = \{ x_1v_1, \dots x_k v_k | x_{1 ... k} \in \RR\}$
	dvs span $S$ er mængden af alle linearkombinationer af v.\\
	I planen $\RR^2$ er spannet hele planen\\

	\begin{example}
		Vi betragter i $\RR^3$ vektorerne\[\begin{pmatrix}a\\2\\3\end{pmatrix}, v_2 = 
		(3,4,5), v_3 = (7,8,9)\] Vektoren $u$ tilhører span(v1,v2,v3) hvis der findes
		koefficineter $x_n$ så det giver $u$\\
		\[ \begin{pmatrix}1&4&6\\2&5&8\\4&6&9\end{pmatrix}\begin{pmatrix}x_1\\x_2\\x_3\end{pmatrix}
		= \begin{pmatrix}1\\1\\1\end{pmatrix}\]
		Gauss jordangiver så til sidst et ligninssystem med en fri variabel. Altså der er
		uendeligt mange løsninger. Altså $u \in span v_1, v_2, v_3$
	\end{example}

	Derfra ser vi følgende. $u$ ligger i spannet hvis\\
	\begin{itemize}
		\item u tilhører spannet.
		\item ligningssystemet $Ax = u$ har mindst en løsning
	\end{itemize}

	\textbf{Theorem 3.2}\\
	Span er et underrum\\
	\textbf{Def 3.4}\\
	Lad $u$ være et underrum af $\RR^n$ og lad $s \subseteq U$ være en endelig delmængde. Man siger at
	$S$ udspænder $U$ hvis span S = U\\

	\section{Lineær uafhængighed}
	Vi har de tre vektorer fra tidligere. De udspænder en 2d plan i $\RR^3$. Men vektor
	$v_3 \in span v_1, v_2$ Altså er $v_3$ overflødig. Derfor er de lineært afhængige\\

	\textbf{Definition 3.5} Lineær uafhængighed\\
	Et set $S = \{ v_1\dots v_k \}$ af vektorer i $\RR^n$ kaldes afhængige hvis den eneste
	løsning til ligningen $x_1v_1 + \dots + x_kv_k = 0$ er $x_1 = x_k = 0$. Ellers er de
	lineært afhængige.
	\begin{example}
		Det kan skrives som 
		\[ \begin{pmatrix}1&4&6\\2&5&8\\4&6&9\end{pmatrix}\begin{pmatrix}x_1\\x_2\\x_3\end{pmatrix}
		= \begin{pmatrix}0\\0\\0\end{pmatrix}\] som på rre har en fri variabel, altså
		har de uendelig mange løsninger så de er \emph{Lineært afhængige}\\

		Bemærk: Hvis $v_4 \in \RR^3$ er en vilkårlig vækrot da er $v1, v_2, v_3, v_4$ også
		lineært afhængige, da vi bare kan bruge den gamle løsning og sætte $x_4 = 0$\\
	\end{example}

	Sætter du det op på totalmatriceform ved du at hvis $rank A < k$ er systemet afhængigt\\

	En konsekvens af det er\\
	\textbf{Theorem 3.5} Lad $n x n$ er invertibel hvis og kun hvis søjlerne er lineært
	uafhængige.\\

	Derudover så:\\
	\textbf{Theorem 3.6} $v_1, \dots, v_k \in \RR^3$ er lineært uafhængige da $k \leq n$ eller
	fem vektorer i $\RR^4$ skal være uafhængige\\

	Derudover siger vi. i et afhængigt. Der findes mindst en vektor $v_i$ i spannet $v_1 , v_k$

	\section{Baser}
	Lad $u$ være et underrum af rn. en endelig delmængde $B \subseteq U$ kaldes en basis for
	$U$ hvis: $B$ er lineært uafhængigt og $span B = U$


	\subsection{Standardbasen for $\RR^n$ har en basis}
	Hvis $U \neq 0$ så har $U$ uendeligt mange forskellige baser\\

	Antallet af vekrorer i en basis for et underrum er entydigt bestemt

	\chapter{09/05}
	\section{Koordinater}
	Man skal have $n$ retninger.\\

	\textbf{Definition 3.7 og Theorem 3.8}\\
	Lad $B = \{b_1, ..., b_k\}$ være en ordnet basis for et underrum $U \subseteq \RR^n$
	Til hvert $v \in U$ findes entydigt bestemte tal $x_1, \dots, x_k \in \RR$ så\\
	\[v = x_1 b_1 + \dots + x_kb_k\] Vektoren med disse tal kaldes koortinaterne for $v$ mht.
	basen $B$ \[ [V]_b = \begin{pmatrix}x_1\\\dots\\x_k\end{pmatrix}\]
	Altså kan vi for ethvert punkt udspændt af $B$ kan vi opskrive det på koordinatform vha. $B$
	Vi kan finde koordinaterne til en vektor $u$ ved at sætte dem op på et ligningssystem Da:
	\[ v = x_1 b_1 + \dots + x_k+b_k \] \[ [B|u] \rightarrow [I_n|V_b]\] 

	\subsection{Basisskiftmatricen}
	Kan omregne fra et koordinatsystem til et andet. Lad der være to ordnede baser $b,c$
	Fra et og samme underrum $u \subseteq \RR$. Basisskiftmatricen fra $c$ til $b$ er med
	notation fra 7.32 defineret ved $P_{b \leftarrow c}$ omreg\\
	Vigtigt: $[v]_b = P_{B \leftarrow C}[v]_c$

	\begin{example}
	Vi betragter to ordnede baser $C, B$\\
	Vi udregner så alle koordinater til kollonner i $C$ og lægger dem i matricen. Derved får vi
	Basisskiftmatricen $P_{B \leftrightarrow C}$\\
	Så hvis vi har en vekrot $[v]_c$ kan vi udtrykke at $P\cdot[v]_c = [v]_b$
	\end{example}
	\subsubsection{Øvelse}
	Vi har to baser: \[ B =  \begin{bmatrix}1&1\\2&3\end{bmatrix} E = \begin{bmatrix}1&0\\0&1
	\end{bmatrix}\] Og en udleveret basisiskiftmatrice \[ P_{B \leftarrow E} = 
\begin{bmatrix}3&-1\\-2&1\end{bmatrix}\]
	Vi vil nu udtrykke $v = \begin{bmatrix}5\\6\end{bmatrix}$ som koordinater i b\\
	Vi Ganger nu vektoren met matricen\\
	\[Pv = \begin{pmatrix}9\\-4\end{pmatrix} \]
	Derfor har vi \[9\begin{bmatrix}1\\2\end{bmatrix} + -4\begin{bmatrix}1\\3\end{bmatrix}\]

	\subsection{Den inverse til basisskift matrice}
	VI betrakter at $P_{B \leftarrow C}$ er invertibel, og dens inverse $P_{b\leftarrow C}^{-1}
	= P_{c \leftarrow b}$\\
	Så for at finde basisskift fra $P_{x \leftarrow E}$ kan vi bare finde den inverse matrice
	$x$

	Vi kigger fra nu af på matrixen \[A = 
	\left(\begin{array}{rrrrr}
	1 & 0 & 1 & 1 & 0 \\
	2 & 1 & 1 & 0 & 5 \\
	3 & 1 & 2 & 3 & 1 \\
	1 & 1 & 0 & 0 & 3
	\end{array}\right)\]

	\section{Nulrum}
	Null space på eng.\\
	Nulrum er hvor det gælder $null A = \{ x \in \RR^n | Ax = 0 \}$

	\section{Søjlerummet}
	Column space. Lad $A$ være en $mxn$ matrix. Søjlerummet af $A$ er $col A = span \{ alle
	søjler i A\}$\\
	Vi kigger på rre for A\[ B = 
	\left(\begin{array}{rrrrr}
	1 & 0 & 1 & 0 & 2 \\
	0 & 1 & -1 & 0 & 1 \\
	0 & 0 & 0 & 1 & -2 \\
	0 & 0 & 0 & 0 & 0
	\end{array}\right) \] Der er dog to afhængige variable. Vi finder de søjler hvor der står
	pivoter, altså vi skal beholde søjle 3 og fem og får
	\[ A' = 
	\left(\begin{array}{rrrrr}
		1 & 0 &  1  \\
		2 & 1 &  0  \\
		3 & 1 &  3  \\
		1 & 1 &  0 
	\end{array}\right)
	\]
	Som er vores basis\\
	Vi ser så at dimensionen af Søjlerummet er 3dimensionelt $dim(col(A)) = rank(A)$\\

	\section{Rækkerum}
	Lad $A$ være en MxN matrix, rækkerummet af $A$ er row $A = span \{ \text{alle }m \text{
	rækkevektorer i} A\}$\\

	Sætter vi på rækkeechelonform. Så kan vi sige at alle nonzero rækker er vores span(som
	vektorer)\\
	Vi får at dimensionen af $row A = rank A$\\
	Vi får at basis er de rækker hvori der befinder sig pivoter

	
	\section{Lineære transformationer}
	\textbf{Lineære transformationer}\\
	En funktion $T: \RR^N \rightarrow \RR^m$ som respekterer plus og Skalarmultiplikation\\

	\textbf{Matrixtransformationer}
	$T_A: \RR^n \rightarrow \RR^m$ givet ved $T_A(x) = Ax$ for alle $x \in \RR^n$\\
	Enhver matrixtransformation er lineær\\
	3.18 Omvendt: Enhver lineær transformation er en matrixtransformation\\

	\begin{example}
		Lad $\theta$ være en vinkel og lad $R_0:R^2 \rightarrow \RR^2$ være den funktion der
		roterer et punkt i planen vinklen $\theta$ mod uret omkring origo. Det er
		geometrisk klart at $R_\theta$ er en lineær transformation. Det at rotere  
		respekterer plus og Skalarmultiplikation. Derfor siger sætning 3.18 at vi har
		en skjult matrice tilsvarer den\\

		Vi tager de to standardvektorer $e_1, e_2$. Roterer vi de to med $\theta$
		\[A_\theta = (R_\theta(e_1) | R_\theta(e_2)) = \begin{pmatrix}cos \theta&-sin\theta\\
		sin\theta&cos\theta\end{pmatrix}\]
	\end{example}
	\begin{example}
		Vi kigger på spejling\\
		Vi ser at \[ S\begin{pmatrix}x\\y\end{pmatrix} = \begin{pmatrix}x\\-y\end{pmatrix}
		= \begin{pmatrix}1&0\\0&-1\end{pmatrix}\begin{pmatrix}x\\y\end{pmatrix}\]
	\end{example}
	\subsection{Sammensætning af lineære transformationer}
	Lad der være givet to lineære transformationer $T(x) = Ax$ og $S(y)=By$\\
	Den sammensatte transformation $S \circ T = S(T(x)) = S(Ax) = B(Ax)) = (BA)x$

	\begin{example}
		Vi betragter nu både spejling og rotation med vinklen 40
		Vi har så $(S\circ R)$ roterer en vector 40 og spejler den\\
		Vi har så $(R\circ S)$ Spejler en vector og roterer med 40\\
	\end{example}

	\section{Kerne}
	Lad kernen af t være den lineære transformation $x\in\RR^n | T(x) = 0$\\
	Kernen er injektiv hcis $ker T = \{0\}$\\
	Kernen er surjektiv hvis alle vektorer bliver ramt af t\\

	Rotationsmatricen er både injektiv og surjektiv.\\
	Injektiv: Det vil sige alle vektorer der roteres og ender i nul er $(0,0)$. Logik though\\
	Surjektiv: Alle vektorer kan roteres over i($Rot_{-40}(Rot_{40}(x)) == x$)\\

	Hvis en lineær transformation både er injektiv og surjektiv(Bijektiv) dette svarer til
	at matricen $A$ er invertibel. Den inverse lineære transformation er da $A^{-1}x$ hvor $A$
	er transformationsmatricen\\

	\chapter{14/05}
	\section{Prikprodukt}
	Prikproduktet for to vektorer $u,v$ er 
	$u_1*v_1 + u_2\bullet g v_2 + \dots + u_n\bullet g v_n = u \bullet g v$
	Vi har $s(u\bullet g v) = su \cdot sv$\\

	\subsection{Norm}
	Normen af en vektor er $||v|| = \sqrt{v\bullet v}$ Se pyth\\

	Regneregler:
	\begin{itemize}
		\item $||v|| \geq 0$ og $||v|| = 0$ netop hvis $v=0$
		\item $||cv|| = |c|||v||$
		\item $|u\bullet v \leq ||u||||v||$ Cauchy Schwarz ulighed
		\item $||u + v|| \leq ||u|| + ||v||$ (Trekantsuligheden)
	\end{itemize}

	\section{enhedsvektor}
	En vektor $u \in \RR$ kaldes en enhedsvektor hvis $||u|| = 1$\\
	For en vilkårlig vektor $u \neq 0$ i $\RR^n$ er den normerede vektor:
	\[u' = \frac{u}{||u||} \]
	

	\section{Euklidisk afstand}
	Den euklidiske afstand mellem to vektorer i $\RR^n$ defineres som:
	\[d(u,v) = ||u-v|| = \sqrt{(u_1 - v_1)^2 + \dots + (u_n-v_n)^2}\]

	\section{Vinklen mellem to vektorer}
	For $u,v$ i $\RR^2$ eller i $\RR^3$ kan man vha.\ cosinusrelationerne vise 
	\[\cos \Theta = \frac{u\bullet v}{||u||||v||}\]

	\section{Ortogonale vektorer og pythagoras}
	For vinklen $\Theta$ mellem to vektorer u og v har man:
	\[ \Theta = 90 \quad \cos \Theta = \frac{u\bullet v}{||u||||v||} = 0 
	\Leftrightarrow u \bullet v = 0\]
	Definition 4.5, to vektorer $u,v$ i $\RR^n$ kaldes orthogonale hvis $u \bullet v = 0$
	Der skrives $u \perp v$.\\

	Theorem x.xx: Gælder det $u\perp v \in \RR^n$ så gælder det at $||u+v||^2 == ||u||^2 + ||v||^2$
	Er to vektorer orthogonale, så gælder det at \[ ||u+v||^2 = ||u||^2 + ||v||^2 \] Da 
	pythagoras sætning kan bruges på trekanten udspændt over de to retvinklede vektorer

	\section{Ortogonal projektion}
	Definition 4.6\\
	Lad $u \neq 0$ være en vektor i $\RR^n$ og lad $U = span u$ være underrummet udspændt af 
	$u$\\
	For enhver vektor $v$ i $\RR^n$ defineres nu:

	\begin{itemize}
		\item Den ortogonale projektion af $v$ på $U$ er givet ved:
			\[ proj_U(v) = \frac{v\bullet u}{||u||^2}u \]

		\item Komponenten af v ortogonal på $U$ er en vektor der går vinkelrat fra $U$ til $v$
		\[comp_U(v) = v - proj_u(v) = v-\frac{v\bullet u}{||u||^2} u\]

		\item Spejlingen af $v$ i $U$ er givet ved:
		\[ refl_U(V) = 2proj_U(v) - v = 2\frac{v\bullet u}{||u||^2}u - v \]
	\end{itemize}

	\subsection{øvelse}
	\[ U = span u = span \begin{pmatrix}1\\1\end{pmatrix}\] og $v =  \begin{pmatrix}1\\3\end{pmatrix}$
	Vi har $u \bullet v = 4$ Vi har $||u|| = \sqrt{2} \Leftrightarrow ||u||^2 = 2$
	\[ proj_U(v) = \frac{4}{2}u = 2u = \begin{pmatrix}2\\2\end{pmatrix}\]
	Vi har
	\[ comp_U (v) = v - proj_U(v) = \begin{pmatrix}-1\\1\end{pmatrix} \]

	%% Creator: Matplotlib, PGF backend
%%
%% To include the figure in your LaTeX document, write
%%   \input{<filename>.pgf}
%%
%% Make sure the required packages are loaded in your preamble
%%   \usepackage{pgf}
%%
%% Figures using additional raster images can only be included by \input if
%% they are in the same directory as the main LaTeX file. For loading figures
%% from other directories you can use the `import` package
%%   \usepackage{import}
%% and then include the figures with
%%   \import{<path to file>}{<filename>.pgf}
%%
%% Matplotlib used the following preamble
%%   \usepackage{fontspec}
%%   \setmainfont{DejaVu Serif}
%%   \setsansfont{DejaVu Sans}
%%   \setmonofont{DejaVu Sans Mono}
%%
\begingroup%
\makeatletter%
\begin{pgfpicture}%
\pgfpathrectangle{\pgfpointorigin}{\pgfqpoint{5.871529in}{3.877584in}}%
\pgfusepath{use as bounding box, clip}%
\begin{pgfscope}%
\pgfsetbuttcap%
\pgfsetmiterjoin%
\definecolor{currentfill}{rgb}{1.000000,1.000000,1.000000}%
\pgfsetfillcolor{currentfill}%
\pgfsetlinewidth{0.000000pt}%
\definecolor{currentstroke}{rgb}{1.000000,1.000000,1.000000}%
\pgfsetstrokecolor{currentstroke}%
\pgfsetdash{}{0pt}%
\pgfpathmoveto{\pgfqpoint{0.000000in}{0.000000in}}%
\pgfpathlineto{\pgfqpoint{5.871529in}{0.000000in}}%
\pgfpathlineto{\pgfqpoint{5.871529in}{3.877584in}}%
\pgfpathlineto{\pgfqpoint{0.000000in}{3.877584in}}%
\pgfpathclose%
\pgfusepath{fill}%
\end{pgfscope}%
\begin{pgfscope}%
\pgfsetbuttcap%
\pgfsetmiterjoin%
\definecolor{currentfill}{rgb}{1.000000,1.000000,1.000000}%
\pgfsetfillcolor{currentfill}%
\pgfsetlinewidth{0.000000pt}%
\definecolor{currentstroke}{rgb}{0.000000,0.000000,0.000000}%
\pgfsetstrokecolor{currentstroke}%
\pgfsetstrokeopacity{0.000000}%
\pgfsetdash{}{0pt}%
\pgfpathmoveto{\pgfqpoint{0.313819in}{0.264754in}}%
\pgfpathlineto{\pgfqpoint{5.736529in}{0.264754in}}%
\pgfpathlineto{\pgfqpoint{5.736529in}{3.742584in}}%
\pgfpathlineto{\pgfqpoint{0.313819in}{3.742584in}}%
\pgfpathclose%
\pgfusepath{fill}%
\end{pgfscope}%
\begin{pgfscope}%
\pgfsetbuttcap%
\pgfsetroundjoin%
\definecolor{currentfill}{rgb}{0.000000,0.000000,0.000000}%
\pgfsetfillcolor{currentfill}%
\pgfsetlinewidth{0.803000pt}%
\definecolor{currentstroke}{rgb}{0.000000,0.000000,0.000000}%
\pgfsetstrokecolor{currentstroke}%
\pgfsetdash{}{0pt}%
\pgfsys@defobject{currentmarker}{\pgfqpoint{0.000000in}{-0.048611in}}{\pgfqpoint{0.000000in}{0.000000in}}{%
\pgfpathmoveto{\pgfqpoint{0.000000in}{0.000000in}}%
\pgfpathlineto{\pgfqpoint{0.000000in}{-0.048611in}}%
\pgfusepath{stroke,fill}%
}%
\begin{pgfscope}%
\pgfsys@transformshift{0.418102in}{0.331635in}%
\pgfsys@useobject{currentmarker}{}%
\end{pgfscope}%
\end{pgfscope}%
\begin{pgfscope}%
\pgfsetbuttcap%
\pgfsetroundjoin%
\definecolor{currentfill}{rgb}{0.000000,0.000000,0.000000}%
\pgfsetfillcolor{currentfill}%
\pgfsetlinewidth{0.803000pt}%
\definecolor{currentstroke}{rgb}{0.000000,0.000000,0.000000}%
\pgfsetstrokecolor{currentstroke}%
\pgfsetdash{}{0pt}%
\pgfsys@defobject{currentmarker}{\pgfqpoint{0.000000in}{-0.048611in}}{\pgfqpoint{0.000000in}{0.000000in}}{%
\pgfpathmoveto{\pgfqpoint{0.000000in}{0.000000in}}%
\pgfpathlineto{\pgfqpoint{0.000000in}{-0.048611in}}%
\pgfusepath{stroke,fill}%
}%
\begin{pgfscope}%
\pgfsys@transformshift{1.287126in}{0.331635in}%
\pgfsys@useobject{currentmarker}{}%
\end{pgfscope}%
\end{pgfscope}%
\begin{pgfscope}%
\pgftext[x=1.287126in,y=0.234413in,,top]{\sffamily\fontsize{10.000000}{12.000000}\selectfont 0.5}%
\end{pgfscope}%
\begin{pgfscope}%
\pgfsetbuttcap%
\pgfsetroundjoin%
\definecolor{currentfill}{rgb}{0.000000,0.000000,0.000000}%
\pgfsetfillcolor{currentfill}%
\pgfsetlinewidth{0.803000pt}%
\definecolor{currentstroke}{rgb}{0.000000,0.000000,0.000000}%
\pgfsetstrokecolor{currentstroke}%
\pgfsetdash{}{0pt}%
\pgfsys@defobject{currentmarker}{\pgfqpoint{0.000000in}{-0.048611in}}{\pgfqpoint{0.000000in}{0.000000in}}{%
\pgfpathmoveto{\pgfqpoint{0.000000in}{0.000000in}}%
\pgfpathlineto{\pgfqpoint{0.000000in}{-0.048611in}}%
\pgfusepath{stroke,fill}%
}%
\begin{pgfscope}%
\pgfsys@transformshift{2.156150in}{0.331635in}%
\pgfsys@useobject{currentmarker}{}%
\end{pgfscope}%
\end{pgfscope}%
\begin{pgfscope}%
\pgftext[x=2.156150in,y=0.234413in,,top]{\sffamily\fontsize{10.000000}{12.000000}\selectfont 1}%
\end{pgfscope}%
\begin{pgfscope}%
\pgfsetbuttcap%
\pgfsetroundjoin%
\definecolor{currentfill}{rgb}{0.000000,0.000000,0.000000}%
\pgfsetfillcolor{currentfill}%
\pgfsetlinewidth{0.803000pt}%
\definecolor{currentstroke}{rgb}{0.000000,0.000000,0.000000}%
\pgfsetstrokecolor{currentstroke}%
\pgfsetdash{}{0pt}%
\pgfsys@defobject{currentmarker}{\pgfqpoint{0.000000in}{-0.048611in}}{\pgfqpoint{0.000000in}{0.000000in}}{%
\pgfpathmoveto{\pgfqpoint{0.000000in}{0.000000in}}%
\pgfpathlineto{\pgfqpoint{0.000000in}{-0.048611in}}%
\pgfusepath{stroke,fill}%
}%
\begin{pgfscope}%
\pgfsys@transformshift{3.025174in}{0.331635in}%
\pgfsys@useobject{currentmarker}{}%
\end{pgfscope}%
\end{pgfscope}%
\begin{pgfscope}%
\pgftext[x=3.025174in,y=0.234413in,,top]{\sffamily\fontsize{10.000000}{12.000000}\selectfont 1.5}%
\end{pgfscope}%
\begin{pgfscope}%
\pgfsetbuttcap%
\pgfsetroundjoin%
\definecolor{currentfill}{rgb}{0.000000,0.000000,0.000000}%
\pgfsetfillcolor{currentfill}%
\pgfsetlinewidth{0.803000pt}%
\definecolor{currentstroke}{rgb}{0.000000,0.000000,0.000000}%
\pgfsetstrokecolor{currentstroke}%
\pgfsetdash{}{0pt}%
\pgfsys@defobject{currentmarker}{\pgfqpoint{0.000000in}{-0.048611in}}{\pgfqpoint{0.000000in}{0.000000in}}{%
\pgfpathmoveto{\pgfqpoint{0.000000in}{0.000000in}}%
\pgfpathlineto{\pgfqpoint{0.000000in}{-0.048611in}}%
\pgfusepath{stroke,fill}%
}%
\begin{pgfscope}%
\pgfsys@transformshift{3.894198in}{0.331635in}%
\pgfsys@useobject{currentmarker}{}%
\end{pgfscope}%
\end{pgfscope}%
\begin{pgfscope}%
\pgftext[x=3.894198in,y=0.234413in,,top]{\sffamily\fontsize{10.000000}{12.000000}\selectfont 2}%
\end{pgfscope}%
\begin{pgfscope}%
\pgfsetbuttcap%
\pgfsetroundjoin%
\definecolor{currentfill}{rgb}{0.000000,0.000000,0.000000}%
\pgfsetfillcolor{currentfill}%
\pgfsetlinewidth{0.803000pt}%
\definecolor{currentstroke}{rgb}{0.000000,0.000000,0.000000}%
\pgfsetstrokecolor{currentstroke}%
\pgfsetdash{}{0pt}%
\pgfsys@defobject{currentmarker}{\pgfqpoint{0.000000in}{-0.048611in}}{\pgfqpoint{0.000000in}{0.000000in}}{%
\pgfpathmoveto{\pgfqpoint{0.000000in}{0.000000in}}%
\pgfpathlineto{\pgfqpoint{0.000000in}{-0.048611in}}%
\pgfusepath{stroke,fill}%
}%
\begin{pgfscope}%
\pgfsys@transformshift{4.763222in}{0.331635in}%
\pgfsys@useobject{currentmarker}{}%
\end{pgfscope}%
\end{pgfscope}%
\begin{pgfscope}%
\pgftext[x=4.763222in,y=0.234413in,,top]{\sffamily\fontsize{10.000000}{12.000000}\selectfont 2.5}%
\end{pgfscope}%
\begin{pgfscope}%
\pgfsetbuttcap%
\pgfsetroundjoin%
\definecolor{currentfill}{rgb}{0.000000,0.000000,0.000000}%
\pgfsetfillcolor{currentfill}%
\pgfsetlinewidth{0.803000pt}%
\definecolor{currentstroke}{rgb}{0.000000,0.000000,0.000000}%
\pgfsetstrokecolor{currentstroke}%
\pgfsetdash{}{0pt}%
\pgfsys@defobject{currentmarker}{\pgfqpoint{0.000000in}{-0.048611in}}{\pgfqpoint{0.000000in}{0.000000in}}{%
\pgfpathmoveto{\pgfqpoint{0.000000in}{0.000000in}}%
\pgfpathlineto{\pgfqpoint{0.000000in}{-0.048611in}}%
\pgfusepath{stroke,fill}%
}%
\begin{pgfscope}%
\pgfsys@transformshift{5.632246in}{0.331635in}%
\pgfsys@useobject{currentmarker}{}%
\end{pgfscope}%
\end{pgfscope}%
\begin{pgfscope}%
\pgftext[x=5.632246in,y=0.234413in,,top]{\sffamily\fontsize{10.000000}{12.000000}\selectfont 3}%
\end{pgfscope}%
\begin{pgfscope}%
\pgfsetbuttcap%
\pgfsetroundjoin%
\definecolor{currentfill}{rgb}{0.000000,0.000000,0.000000}%
\pgfsetfillcolor{currentfill}%
\pgfsetlinewidth{0.602250pt}%
\definecolor{currentstroke}{rgb}{0.000000,0.000000,0.000000}%
\pgfsetstrokecolor{currentstroke}%
\pgfsetdash{}{0pt}%
\pgfsys@defobject{currentmarker}{\pgfqpoint{0.000000in}{-0.027778in}}{\pgfqpoint{0.000000in}{0.000000in}}{%
\pgfpathmoveto{\pgfqpoint{0.000000in}{0.000000in}}%
\pgfpathlineto{\pgfqpoint{0.000000in}{-0.027778in}}%
\pgfusepath{stroke,fill}%
}%
\begin{pgfscope}%
\pgfsys@transformshift{0.591906in}{0.331635in}%
\pgfsys@useobject{currentmarker}{}%
\end{pgfscope}%
\end{pgfscope}%
\begin{pgfscope}%
\pgfsetbuttcap%
\pgfsetroundjoin%
\definecolor{currentfill}{rgb}{0.000000,0.000000,0.000000}%
\pgfsetfillcolor{currentfill}%
\pgfsetlinewidth{0.602250pt}%
\definecolor{currentstroke}{rgb}{0.000000,0.000000,0.000000}%
\pgfsetstrokecolor{currentstroke}%
\pgfsetdash{}{0pt}%
\pgfsys@defobject{currentmarker}{\pgfqpoint{0.000000in}{-0.027778in}}{\pgfqpoint{0.000000in}{0.000000in}}{%
\pgfpathmoveto{\pgfqpoint{0.000000in}{0.000000in}}%
\pgfpathlineto{\pgfqpoint{0.000000in}{-0.027778in}}%
\pgfusepath{stroke,fill}%
}%
\begin{pgfscope}%
\pgfsys@transformshift{0.765711in}{0.331635in}%
\pgfsys@useobject{currentmarker}{}%
\end{pgfscope}%
\end{pgfscope}%
\begin{pgfscope}%
\pgfsetbuttcap%
\pgfsetroundjoin%
\definecolor{currentfill}{rgb}{0.000000,0.000000,0.000000}%
\pgfsetfillcolor{currentfill}%
\pgfsetlinewidth{0.602250pt}%
\definecolor{currentstroke}{rgb}{0.000000,0.000000,0.000000}%
\pgfsetstrokecolor{currentstroke}%
\pgfsetdash{}{0pt}%
\pgfsys@defobject{currentmarker}{\pgfqpoint{0.000000in}{-0.027778in}}{\pgfqpoint{0.000000in}{0.000000in}}{%
\pgfpathmoveto{\pgfqpoint{0.000000in}{0.000000in}}%
\pgfpathlineto{\pgfqpoint{0.000000in}{-0.027778in}}%
\pgfusepath{stroke,fill}%
}%
\begin{pgfscope}%
\pgfsys@transformshift{0.939516in}{0.331635in}%
\pgfsys@useobject{currentmarker}{}%
\end{pgfscope}%
\end{pgfscope}%
\begin{pgfscope}%
\pgfsetbuttcap%
\pgfsetroundjoin%
\definecolor{currentfill}{rgb}{0.000000,0.000000,0.000000}%
\pgfsetfillcolor{currentfill}%
\pgfsetlinewidth{0.602250pt}%
\definecolor{currentstroke}{rgb}{0.000000,0.000000,0.000000}%
\pgfsetstrokecolor{currentstroke}%
\pgfsetdash{}{0pt}%
\pgfsys@defobject{currentmarker}{\pgfqpoint{0.000000in}{-0.027778in}}{\pgfqpoint{0.000000in}{0.000000in}}{%
\pgfpathmoveto{\pgfqpoint{0.000000in}{0.000000in}}%
\pgfpathlineto{\pgfqpoint{0.000000in}{-0.027778in}}%
\pgfusepath{stroke,fill}%
}%
\begin{pgfscope}%
\pgfsys@transformshift{1.113321in}{0.331635in}%
\pgfsys@useobject{currentmarker}{}%
\end{pgfscope}%
\end{pgfscope}%
\begin{pgfscope}%
\pgfsetbuttcap%
\pgfsetroundjoin%
\definecolor{currentfill}{rgb}{0.000000,0.000000,0.000000}%
\pgfsetfillcolor{currentfill}%
\pgfsetlinewidth{0.602250pt}%
\definecolor{currentstroke}{rgb}{0.000000,0.000000,0.000000}%
\pgfsetstrokecolor{currentstroke}%
\pgfsetdash{}{0pt}%
\pgfsys@defobject{currentmarker}{\pgfqpoint{0.000000in}{-0.027778in}}{\pgfqpoint{0.000000in}{0.000000in}}{%
\pgfpathmoveto{\pgfqpoint{0.000000in}{0.000000in}}%
\pgfpathlineto{\pgfqpoint{0.000000in}{-0.027778in}}%
\pgfusepath{stroke,fill}%
}%
\begin{pgfscope}%
\pgfsys@transformshift{1.460931in}{0.331635in}%
\pgfsys@useobject{currentmarker}{}%
\end{pgfscope}%
\end{pgfscope}%
\begin{pgfscope}%
\pgfsetbuttcap%
\pgfsetroundjoin%
\definecolor{currentfill}{rgb}{0.000000,0.000000,0.000000}%
\pgfsetfillcolor{currentfill}%
\pgfsetlinewidth{0.602250pt}%
\definecolor{currentstroke}{rgb}{0.000000,0.000000,0.000000}%
\pgfsetstrokecolor{currentstroke}%
\pgfsetdash{}{0pt}%
\pgfsys@defobject{currentmarker}{\pgfqpoint{0.000000in}{-0.027778in}}{\pgfqpoint{0.000000in}{0.000000in}}{%
\pgfpathmoveto{\pgfqpoint{0.000000in}{0.000000in}}%
\pgfpathlineto{\pgfqpoint{0.000000in}{-0.027778in}}%
\pgfusepath{stroke,fill}%
}%
\begin{pgfscope}%
\pgfsys@transformshift{1.634735in}{0.331635in}%
\pgfsys@useobject{currentmarker}{}%
\end{pgfscope}%
\end{pgfscope}%
\begin{pgfscope}%
\pgfsetbuttcap%
\pgfsetroundjoin%
\definecolor{currentfill}{rgb}{0.000000,0.000000,0.000000}%
\pgfsetfillcolor{currentfill}%
\pgfsetlinewidth{0.602250pt}%
\definecolor{currentstroke}{rgb}{0.000000,0.000000,0.000000}%
\pgfsetstrokecolor{currentstroke}%
\pgfsetdash{}{0pt}%
\pgfsys@defobject{currentmarker}{\pgfqpoint{0.000000in}{-0.027778in}}{\pgfqpoint{0.000000in}{0.000000in}}{%
\pgfpathmoveto{\pgfqpoint{0.000000in}{0.000000in}}%
\pgfpathlineto{\pgfqpoint{0.000000in}{-0.027778in}}%
\pgfusepath{stroke,fill}%
}%
\begin{pgfscope}%
\pgfsys@transformshift{1.808540in}{0.331635in}%
\pgfsys@useobject{currentmarker}{}%
\end{pgfscope}%
\end{pgfscope}%
\begin{pgfscope}%
\pgfsetbuttcap%
\pgfsetroundjoin%
\definecolor{currentfill}{rgb}{0.000000,0.000000,0.000000}%
\pgfsetfillcolor{currentfill}%
\pgfsetlinewidth{0.602250pt}%
\definecolor{currentstroke}{rgb}{0.000000,0.000000,0.000000}%
\pgfsetstrokecolor{currentstroke}%
\pgfsetdash{}{0pt}%
\pgfsys@defobject{currentmarker}{\pgfqpoint{0.000000in}{-0.027778in}}{\pgfqpoint{0.000000in}{0.000000in}}{%
\pgfpathmoveto{\pgfqpoint{0.000000in}{0.000000in}}%
\pgfpathlineto{\pgfqpoint{0.000000in}{-0.027778in}}%
\pgfusepath{stroke,fill}%
}%
\begin{pgfscope}%
\pgfsys@transformshift{1.982345in}{0.331635in}%
\pgfsys@useobject{currentmarker}{}%
\end{pgfscope}%
\end{pgfscope}%
\begin{pgfscope}%
\pgfsetbuttcap%
\pgfsetroundjoin%
\definecolor{currentfill}{rgb}{0.000000,0.000000,0.000000}%
\pgfsetfillcolor{currentfill}%
\pgfsetlinewidth{0.602250pt}%
\definecolor{currentstroke}{rgb}{0.000000,0.000000,0.000000}%
\pgfsetstrokecolor{currentstroke}%
\pgfsetdash{}{0pt}%
\pgfsys@defobject{currentmarker}{\pgfqpoint{0.000000in}{-0.027778in}}{\pgfqpoint{0.000000in}{0.000000in}}{%
\pgfpathmoveto{\pgfqpoint{0.000000in}{0.000000in}}%
\pgfpathlineto{\pgfqpoint{0.000000in}{-0.027778in}}%
\pgfusepath{stroke,fill}%
}%
\begin{pgfscope}%
\pgfsys@transformshift{2.329955in}{0.331635in}%
\pgfsys@useobject{currentmarker}{}%
\end{pgfscope}%
\end{pgfscope}%
\begin{pgfscope}%
\pgfsetbuttcap%
\pgfsetroundjoin%
\definecolor{currentfill}{rgb}{0.000000,0.000000,0.000000}%
\pgfsetfillcolor{currentfill}%
\pgfsetlinewidth{0.602250pt}%
\definecolor{currentstroke}{rgb}{0.000000,0.000000,0.000000}%
\pgfsetstrokecolor{currentstroke}%
\pgfsetdash{}{0pt}%
\pgfsys@defobject{currentmarker}{\pgfqpoint{0.000000in}{-0.027778in}}{\pgfqpoint{0.000000in}{0.000000in}}{%
\pgfpathmoveto{\pgfqpoint{0.000000in}{0.000000in}}%
\pgfpathlineto{\pgfqpoint{0.000000in}{-0.027778in}}%
\pgfusepath{stroke,fill}%
}%
\begin{pgfscope}%
\pgfsys@transformshift{2.503759in}{0.331635in}%
\pgfsys@useobject{currentmarker}{}%
\end{pgfscope}%
\end{pgfscope}%
\begin{pgfscope}%
\pgfsetbuttcap%
\pgfsetroundjoin%
\definecolor{currentfill}{rgb}{0.000000,0.000000,0.000000}%
\pgfsetfillcolor{currentfill}%
\pgfsetlinewidth{0.602250pt}%
\definecolor{currentstroke}{rgb}{0.000000,0.000000,0.000000}%
\pgfsetstrokecolor{currentstroke}%
\pgfsetdash{}{0pt}%
\pgfsys@defobject{currentmarker}{\pgfqpoint{0.000000in}{-0.027778in}}{\pgfqpoint{0.000000in}{0.000000in}}{%
\pgfpathmoveto{\pgfqpoint{0.000000in}{0.000000in}}%
\pgfpathlineto{\pgfqpoint{0.000000in}{-0.027778in}}%
\pgfusepath{stroke,fill}%
}%
\begin{pgfscope}%
\pgfsys@transformshift{2.677564in}{0.331635in}%
\pgfsys@useobject{currentmarker}{}%
\end{pgfscope}%
\end{pgfscope}%
\begin{pgfscope}%
\pgfsetbuttcap%
\pgfsetroundjoin%
\definecolor{currentfill}{rgb}{0.000000,0.000000,0.000000}%
\pgfsetfillcolor{currentfill}%
\pgfsetlinewidth{0.602250pt}%
\definecolor{currentstroke}{rgb}{0.000000,0.000000,0.000000}%
\pgfsetstrokecolor{currentstroke}%
\pgfsetdash{}{0pt}%
\pgfsys@defobject{currentmarker}{\pgfqpoint{0.000000in}{-0.027778in}}{\pgfqpoint{0.000000in}{0.000000in}}{%
\pgfpathmoveto{\pgfqpoint{0.000000in}{0.000000in}}%
\pgfpathlineto{\pgfqpoint{0.000000in}{-0.027778in}}%
\pgfusepath{stroke,fill}%
}%
\begin{pgfscope}%
\pgfsys@transformshift{2.851369in}{0.331635in}%
\pgfsys@useobject{currentmarker}{}%
\end{pgfscope}%
\end{pgfscope}%
\begin{pgfscope}%
\pgfsetbuttcap%
\pgfsetroundjoin%
\definecolor{currentfill}{rgb}{0.000000,0.000000,0.000000}%
\pgfsetfillcolor{currentfill}%
\pgfsetlinewidth{0.602250pt}%
\definecolor{currentstroke}{rgb}{0.000000,0.000000,0.000000}%
\pgfsetstrokecolor{currentstroke}%
\pgfsetdash{}{0pt}%
\pgfsys@defobject{currentmarker}{\pgfqpoint{0.000000in}{-0.027778in}}{\pgfqpoint{0.000000in}{0.000000in}}{%
\pgfpathmoveto{\pgfqpoint{0.000000in}{0.000000in}}%
\pgfpathlineto{\pgfqpoint{0.000000in}{-0.027778in}}%
\pgfusepath{stroke,fill}%
}%
\begin{pgfscope}%
\pgfsys@transformshift{3.198979in}{0.331635in}%
\pgfsys@useobject{currentmarker}{}%
\end{pgfscope}%
\end{pgfscope}%
\begin{pgfscope}%
\pgfsetbuttcap%
\pgfsetroundjoin%
\definecolor{currentfill}{rgb}{0.000000,0.000000,0.000000}%
\pgfsetfillcolor{currentfill}%
\pgfsetlinewidth{0.602250pt}%
\definecolor{currentstroke}{rgb}{0.000000,0.000000,0.000000}%
\pgfsetstrokecolor{currentstroke}%
\pgfsetdash{}{0pt}%
\pgfsys@defobject{currentmarker}{\pgfqpoint{0.000000in}{-0.027778in}}{\pgfqpoint{0.000000in}{0.000000in}}{%
\pgfpathmoveto{\pgfqpoint{0.000000in}{0.000000in}}%
\pgfpathlineto{\pgfqpoint{0.000000in}{-0.027778in}}%
\pgfusepath{stroke,fill}%
}%
\begin{pgfscope}%
\pgfsys@transformshift{3.372784in}{0.331635in}%
\pgfsys@useobject{currentmarker}{}%
\end{pgfscope}%
\end{pgfscope}%
\begin{pgfscope}%
\pgfsetbuttcap%
\pgfsetroundjoin%
\definecolor{currentfill}{rgb}{0.000000,0.000000,0.000000}%
\pgfsetfillcolor{currentfill}%
\pgfsetlinewidth{0.602250pt}%
\definecolor{currentstroke}{rgb}{0.000000,0.000000,0.000000}%
\pgfsetstrokecolor{currentstroke}%
\pgfsetdash{}{0pt}%
\pgfsys@defobject{currentmarker}{\pgfqpoint{0.000000in}{-0.027778in}}{\pgfqpoint{0.000000in}{0.000000in}}{%
\pgfpathmoveto{\pgfqpoint{0.000000in}{0.000000in}}%
\pgfpathlineto{\pgfqpoint{0.000000in}{-0.027778in}}%
\pgfusepath{stroke,fill}%
}%
\begin{pgfscope}%
\pgfsys@transformshift{3.546588in}{0.331635in}%
\pgfsys@useobject{currentmarker}{}%
\end{pgfscope}%
\end{pgfscope}%
\begin{pgfscope}%
\pgfsetbuttcap%
\pgfsetroundjoin%
\definecolor{currentfill}{rgb}{0.000000,0.000000,0.000000}%
\pgfsetfillcolor{currentfill}%
\pgfsetlinewidth{0.602250pt}%
\definecolor{currentstroke}{rgb}{0.000000,0.000000,0.000000}%
\pgfsetstrokecolor{currentstroke}%
\pgfsetdash{}{0pt}%
\pgfsys@defobject{currentmarker}{\pgfqpoint{0.000000in}{-0.027778in}}{\pgfqpoint{0.000000in}{0.000000in}}{%
\pgfpathmoveto{\pgfqpoint{0.000000in}{0.000000in}}%
\pgfpathlineto{\pgfqpoint{0.000000in}{-0.027778in}}%
\pgfusepath{stroke,fill}%
}%
\begin{pgfscope}%
\pgfsys@transformshift{3.720393in}{0.331635in}%
\pgfsys@useobject{currentmarker}{}%
\end{pgfscope}%
\end{pgfscope}%
\begin{pgfscope}%
\pgfsetbuttcap%
\pgfsetroundjoin%
\definecolor{currentfill}{rgb}{0.000000,0.000000,0.000000}%
\pgfsetfillcolor{currentfill}%
\pgfsetlinewidth{0.602250pt}%
\definecolor{currentstroke}{rgb}{0.000000,0.000000,0.000000}%
\pgfsetstrokecolor{currentstroke}%
\pgfsetdash{}{0pt}%
\pgfsys@defobject{currentmarker}{\pgfqpoint{0.000000in}{-0.027778in}}{\pgfqpoint{0.000000in}{0.000000in}}{%
\pgfpathmoveto{\pgfqpoint{0.000000in}{0.000000in}}%
\pgfpathlineto{\pgfqpoint{0.000000in}{-0.027778in}}%
\pgfusepath{stroke,fill}%
}%
\begin{pgfscope}%
\pgfsys@transformshift{4.068003in}{0.331635in}%
\pgfsys@useobject{currentmarker}{}%
\end{pgfscope}%
\end{pgfscope}%
\begin{pgfscope}%
\pgfsetbuttcap%
\pgfsetroundjoin%
\definecolor{currentfill}{rgb}{0.000000,0.000000,0.000000}%
\pgfsetfillcolor{currentfill}%
\pgfsetlinewidth{0.602250pt}%
\definecolor{currentstroke}{rgb}{0.000000,0.000000,0.000000}%
\pgfsetstrokecolor{currentstroke}%
\pgfsetdash{}{0pt}%
\pgfsys@defobject{currentmarker}{\pgfqpoint{0.000000in}{-0.027778in}}{\pgfqpoint{0.000000in}{0.000000in}}{%
\pgfpathmoveto{\pgfqpoint{0.000000in}{0.000000in}}%
\pgfpathlineto{\pgfqpoint{0.000000in}{-0.027778in}}%
\pgfusepath{stroke,fill}%
}%
\begin{pgfscope}%
\pgfsys@transformshift{4.241808in}{0.331635in}%
\pgfsys@useobject{currentmarker}{}%
\end{pgfscope}%
\end{pgfscope}%
\begin{pgfscope}%
\pgfsetbuttcap%
\pgfsetroundjoin%
\definecolor{currentfill}{rgb}{0.000000,0.000000,0.000000}%
\pgfsetfillcolor{currentfill}%
\pgfsetlinewidth{0.602250pt}%
\definecolor{currentstroke}{rgb}{0.000000,0.000000,0.000000}%
\pgfsetstrokecolor{currentstroke}%
\pgfsetdash{}{0pt}%
\pgfsys@defobject{currentmarker}{\pgfqpoint{0.000000in}{-0.027778in}}{\pgfqpoint{0.000000in}{0.000000in}}{%
\pgfpathmoveto{\pgfqpoint{0.000000in}{0.000000in}}%
\pgfpathlineto{\pgfqpoint{0.000000in}{-0.027778in}}%
\pgfusepath{stroke,fill}%
}%
\begin{pgfscope}%
\pgfsys@transformshift{4.415612in}{0.331635in}%
\pgfsys@useobject{currentmarker}{}%
\end{pgfscope}%
\end{pgfscope}%
\begin{pgfscope}%
\pgfsetbuttcap%
\pgfsetroundjoin%
\definecolor{currentfill}{rgb}{0.000000,0.000000,0.000000}%
\pgfsetfillcolor{currentfill}%
\pgfsetlinewidth{0.602250pt}%
\definecolor{currentstroke}{rgb}{0.000000,0.000000,0.000000}%
\pgfsetstrokecolor{currentstroke}%
\pgfsetdash{}{0pt}%
\pgfsys@defobject{currentmarker}{\pgfqpoint{0.000000in}{-0.027778in}}{\pgfqpoint{0.000000in}{0.000000in}}{%
\pgfpathmoveto{\pgfqpoint{0.000000in}{0.000000in}}%
\pgfpathlineto{\pgfqpoint{0.000000in}{-0.027778in}}%
\pgfusepath{stroke,fill}%
}%
\begin{pgfscope}%
\pgfsys@transformshift{4.589417in}{0.331635in}%
\pgfsys@useobject{currentmarker}{}%
\end{pgfscope}%
\end{pgfscope}%
\begin{pgfscope}%
\pgfsetbuttcap%
\pgfsetroundjoin%
\definecolor{currentfill}{rgb}{0.000000,0.000000,0.000000}%
\pgfsetfillcolor{currentfill}%
\pgfsetlinewidth{0.602250pt}%
\definecolor{currentstroke}{rgb}{0.000000,0.000000,0.000000}%
\pgfsetstrokecolor{currentstroke}%
\pgfsetdash{}{0pt}%
\pgfsys@defobject{currentmarker}{\pgfqpoint{0.000000in}{-0.027778in}}{\pgfqpoint{0.000000in}{0.000000in}}{%
\pgfpathmoveto{\pgfqpoint{0.000000in}{0.000000in}}%
\pgfpathlineto{\pgfqpoint{0.000000in}{-0.027778in}}%
\pgfusepath{stroke,fill}%
}%
\begin{pgfscope}%
\pgfsys@transformshift{4.937027in}{0.331635in}%
\pgfsys@useobject{currentmarker}{}%
\end{pgfscope}%
\end{pgfscope}%
\begin{pgfscope}%
\pgfsetbuttcap%
\pgfsetroundjoin%
\definecolor{currentfill}{rgb}{0.000000,0.000000,0.000000}%
\pgfsetfillcolor{currentfill}%
\pgfsetlinewidth{0.602250pt}%
\definecolor{currentstroke}{rgb}{0.000000,0.000000,0.000000}%
\pgfsetstrokecolor{currentstroke}%
\pgfsetdash{}{0pt}%
\pgfsys@defobject{currentmarker}{\pgfqpoint{0.000000in}{-0.027778in}}{\pgfqpoint{0.000000in}{0.000000in}}{%
\pgfpathmoveto{\pgfqpoint{0.000000in}{0.000000in}}%
\pgfpathlineto{\pgfqpoint{0.000000in}{-0.027778in}}%
\pgfusepath{stroke,fill}%
}%
\begin{pgfscope}%
\pgfsys@transformshift{5.110832in}{0.331635in}%
\pgfsys@useobject{currentmarker}{}%
\end{pgfscope}%
\end{pgfscope}%
\begin{pgfscope}%
\pgfsetbuttcap%
\pgfsetroundjoin%
\definecolor{currentfill}{rgb}{0.000000,0.000000,0.000000}%
\pgfsetfillcolor{currentfill}%
\pgfsetlinewidth{0.602250pt}%
\definecolor{currentstroke}{rgb}{0.000000,0.000000,0.000000}%
\pgfsetstrokecolor{currentstroke}%
\pgfsetdash{}{0pt}%
\pgfsys@defobject{currentmarker}{\pgfqpoint{0.000000in}{-0.027778in}}{\pgfqpoint{0.000000in}{0.000000in}}{%
\pgfpathmoveto{\pgfqpoint{0.000000in}{0.000000in}}%
\pgfpathlineto{\pgfqpoint{0.000000in}{-0.027778in}}%
\pgfusepath{stroke,fill}%
}%
\begin{pgfscope}%
\pgfsys@transformshift{5.284637in}{0.331635in}%
\pgfsys@useobject{currentmarker}{}%
\end{pgfscope}%
\end{pgfscope}%
\begin{pgfscope}%
\pgfsetbuttcap%
\pgfsetroundjoin%
\definecolor{currentfill}{rgb}{0.000000,0.000000,0.000000}%
\pgfsetfillcolor{currentfill}%
\pgfsetlinewidth{0.602250pt}%
\definecolor{currentstroke}{rgb}{0.000000,0.000000,0.000000}%
\pgfsetstrokecolor{currentstroke}%
\pgfsetdash{}{0pt}%
\pgfsys@defobject{currentmarker}{\pgfqpoint{0.000000in}{-0.027778in}}{\pgfqpoint{0.000000in}{0.000000in}}{%
\pgfpathmoveto{\pgfqpoint{0.000000in}{0.000000in}}%
\pgfpathlineto{\pgfqpoint{0.000000in}{-0.027778in}}%
\pgfusepath{stroke,fill}%
}%
\begin{pgfscope}%
\pgfsys@transformshift{5.458441in}{0.331635in}%
\pgfsys@useobject{currentmarker}{}%
\end{pgfscope}%
\end{pgfscope}%
\begin{pgfscope}%
\pgfsetbuttcap%
\pgfsetroundjoin%
\definecolor{currentfill}{rgb}{0.000000,0.000000,0.000000}%
\pgfsetfillcolor{currentfill}%
\pgfsetlinewidth{0.803000pt}%
\definecolor{currentstroke}{rgb}{0.000000,0.000000,0.000000}%
\pgfsetstrokecolor{currentstroke}%
\pgfsetdash{}{0pt}%
\pgfsys@defobject{currentmarker}{\pgfqpoint{-0.048611in}{0.000000in}}{\pgfqpoint{0.000000in}{0.000000in}}{%
\pgfpathmoveto{\pgfqpoint{0.000000in}{0.000000in}}%
\pgfpathlineto{\pgfqpoint{-0.048611in}{0.000000in}}%
\pgfusepath{stroke,fill}%
}%
\begin{pgfscope}%
\pgfsys@transformshift{0.418102in}{0.331635in}%
\pgfsys@useobject{currentmarker}{}%
\end{pgfscope}%
\end{pgfscope}%
\begin{pgfscope}%
\pgfsetbuttcap%
\pgfsetroundjoin%
\definecolor{currentfill}{rgb}{0.000000,0.000000,0.000000}%
\pgfsetfillcolor{currentfill}%
\pgfsetlinewidth{0.803000pt}%
\definecolor{currentstroke}{rgb}{0.000000,0.000000,0.000000}%
\pgfsetstrokecolor{currentstroke}%
\pgfsetdash{}{0pt}%
\pgfsys@defobject{currentmarker}{\pgfqpoint{-0.048611in}{0.000000in}}{\pgfqpoint{0.000000in}{0.000000in}}{%
\pgfpathmoveto{\pgfqpoint{0.000000in}{0.000000in}}%
\pgfpathlineto{\pgfqpoint{-0.048611in}{0.000000in}}%
\pgfusepath{stroke,fill}%
}%
\begin{pgfscope}%
\pgfsys@transformshift{0.418102in}{0.888980in}%
\pgfsys@useobject{currentmarker}{}%
\end{pgfscope}%
\end{pgfscope}%
\begin{pgfscope}%
\pgftext[x=0.100000in,y=0.836218in,left,base]{\sffamily\fontsize{10.000000}{12.000000}\selectfont 0.5}%
\end{pgfscope}%
\begin{pgfscope}%
\pgfsetbuttcap%
\pgfsetroundjoin%
\definecolor{currentfill}{rgb}{0.000000,0.000000,0.000000}%
\pgfsetfillcolor{currentfill}%
\pgfsetlinewidth{0.803000pt}%
\definecolor{currentstroke}{rgb}{0.000000,0.000000,0.000000}%
\pgfsetstrokecolor{currentstroke}%
\pgfsetdash{}{0pt}%
\pgfsys@defobject{currentmarker}{\pgfqpoint{-0.048611in}{0.000000in}}{\pgfqpoint{0.000000in}{0.000000in}}{%
\pgfpathmoveto{\pgfqpoint{0.000000in}{0.000000in}}%
\pgfpathlineto{\pgfqpoint{-0.048611in}{0.000000in}}%
\pgfusepath{stroke,fill}%
}%
\begin{pgfscope}%
\pgfsys@transformshift{0.418102in}{1.446324in}%
\pgfsys@useobject{currentmarker}{}%
\end{pgfscope}%
\end{pgfscope}%
\begin{pgfscope}%
\pgftext[x=0.232514in,y=1.393563in,left,base]{\sffamily\fontsize{10.000000}{12.000000}\selectfont 1}%
\end{pgfscope}%
\begin{pgfscope}%
\pgfsetbuttcap%
\pgfsetroundjoin%
\definecolor{currentfill}{rgb}{0.000000,0.000000,0.000000}%
\pgfsetfillcolor{currentfill}%
\pgfsetlinewidth{0.803000pt}%
\definecolor{currentstroke}{rgb}{0.000000,0.000000,0.000000}%
\pgfsetstrokecolor{currentstroke}%
\pgfsetdash{}{0pt}%
\pgfsys@defobject{currentmarker}{\pgfqpoint{-0.048611in}{0.000000in}}{\pgfqpoint{0.000000in}{0.000000in}}{%
\pgfpathmoveto{\pgfqpoint{0.000000in}{0.000000in}}%
\pgfpathlineto{\pgfqpoint{-0.048611in}{0.000000in}}%
\pgfusepath{stroke,fill}%
}%
\begin{pgfscope}%
\pgfsys@transformshift{0.418102in}{2.003669in}%
\pgfsys@useobject{currentmarker}{}%
\end{pgfscope}%
\end{pgfscope}%
\begin{pgfscope}%
\pgftext[x=0.100000in,y=1.950908in,left,base]{\sffamily\fontsize{10.000000}{12.000000}\selectfont 1.5}%
\end{pgfscope}%
\begin{pgfscope}%
\pgfsetbuttcap%
\pgfsetroundjoin%
\definecolor{currentfill}{rgb}{0.000000,0.000000,0.000000}%
\pgfsetfillcolor{currentfill}%
\pgfsetlinewidth{0.803000pt}%
\definecolor{currentstroke}{rgb}{0.000000,0.000000,0.000000}%
\pgfsetstrokecolor{currentstroke}%
\pgfsetdash{}{0pt}%
\pgfsys@defobject{currentmarker}{\pgfqpoint{-0.048611in}{0.000000in}}{\pgfqpoint{0.000000in}{0.000000in}}{%
\pgfpathmoveto{\pgfqpoint{0.000000in}{0.000000in}}%
\pgfpathlineto{\pgfqpoint{-0.048611in}{0.000000in}}%
\pgfusepath{stroke,fill}%
}%
\begin{pgfscope}%
\pgfsys@transformshift{0.418102in}{2.561014in}%
\pgfsys@useobject{currentmarker}{}%
\end{pgfscope}%
\end{pgfscope}%
\begin{pgfscope}%
\pgftext[x=0.232514in,y=2.508252in,left,base]{\sffamily\fontsize{10.000000}{12.000000}\selectfont 2}%
\end{pgfscope}%
\begin{pgfscope}%
\pgfsetbuttcap%
\pgfsetroundjoin%
\definecolor{currentfill}{rgb}{0.000000,0.000000,0.000000}%
\pgfsetfillcolor{currentfill}%
\pgfsetlinewidth{0.803000pt}%
\definecolor{currentstroke}{rgb}{0.000000,0.000000,0.000000}%
\pgfsetstrokecolor{currentstroke}%
\pgfsetdash{}{0pt}%
\pgfsys@defobject{currentmarker}{\pgfqpoint{-0.048611in}{0.000000in}}{\pgfqpoint{0.000000in}{0.000000in}}{%
\pgfpathmoveto{\pgfqpoint{0.000000in}{0.000000in}}%
\pgfpathlineto{\pgfqpoint{-0.048611in}{0.000000in}}%
\pgfusepath{stroke,fill}%
}%
\begin{pgfscope}%
\pgfsys@transformshift{0.418102in}{3.118358in}%
\pgfsys@useobject{currentmarker}{}%
\end{pgfscope}%
\end{pgfscope}%
\begin{pgfscope}%
\pgftext[x=0.100000in,y=3.065597in,left,base]{\sffamily\fontsize{10.000000}{12.000000}\selectfont 2.5}%
\end{pgfscope}%
\begin{pgfscope}%
\pgfsetbuttcap%
\pgfsetroundjoin%
\definecolor{currentfill}{rgb}{0.000000,0.000000,0.000000}%
\pgfsetfillcolor{currentfill}%
\pgfsetlinewidth{0.803000pt}%
\definecolor{currentstroke}{rgb}{0.000000,0.000000,0.000000}%
\pgfsetstrokecolor{currentstroke}%
\pgfsetdash{}{0pt}%
\pgfsys@defobject{currentmarker}{\pgfqpoint{-0.048611in}{0.000000in}}{\pgfqpoint{0.000000in}{0.000000in}}{%
\pgfpathmoveto{\pgfqpoint{0.000000in}{0.000000in}}%
\pgfpathlineto{\pgfqpoint{-0.048611in}{0.000000in}}%
\pgfusepath{stroke,fill}%
}%
\begin{pgfscope}%
\pgfsys@transformshift{0.418102in}{3.675703in}%
\pgfsys@useobject{currentmarker}{}%
\end{pgfscope}%
\end{pgfscope}%
\begin{pgfscope}%
\pgftext[x=0.232514in,y=3.622942in,left,base]{\sffamily\fontsize{10.000000}{12.000000}\selectfont 3}%
\end{pgfscope}%
\begin{pgfscope}%
\pgfsetbuttcap%
\pgfsetroundjoin%
\definecolor{currentfill}{rgb}{0.000000,0.000000,0.000000}%
\pgfsetfillcolor{currentfill}%
\pgfsetlinewidth{0.602250pt}%
\definecolor{currentstroke}{rgb}{0.000000,0.000000,0.000000}%
\pgfsetstrokecolor{currentstroke}%
\pgfsetdash{}{0pt}%
\pgfsys@defobject{currentmarker}{\pgfqpoint{-0.027778in}{0.000000in}}{\pgfqpoint{0.000000in}{0.000000in}}{%
\pgfpathmoveto{\pgfqpoint{0.000000in}{0.000000in}}%
\pgfpathlineto{\pgfqpoint{-0.027778in}{0.000000in}}%
\pgfusepath{stroke,fill}%
}%
\begin{pgfscope}%
\pgfsys@transformshift{0.418102in}{0.443104in}%
\pgfsys@useobject{currentmarker}{}%
\end{pgfscope}%
\end{pgfscope}%
\begin{pgfscope}%
\pgfsetbuttcap%
\pgfsetroundjoin%
\definecolor{currentfill}{rgb}{0.000000,0.000000,0.000000}%
\pgfsetfillcolor{currentfill}%
\pgfsetlinewidth{0.602250pt}%
\definecolor{currentstroke}{rgb}{0.000000,0.000000,0.000000}%
\pgfsetstrokecolor{currentstroke}%
\pgfsetdash{}{0pt}%
\pgfsys@defobject{currentmarker}{\pgfqpoint{-0.027778in}{0.000000in}}{\pgfqpoint{0.000000in}{0.000000in}}{%
\pgfpathmoveto{\pgfqpoint{0.000000in}{0.000000in}}%
\pgfpathlineto{\pgfqpoint{-0.027778in}{0.000000in}}%
\pgfusepath{stroke,fill}%
}%
\begin{pgfscope}%
\pgfsys@transformshift{0.418102in}{0.554573in}%
\pgfsys@useobject{currentmarker}{}%
\end{pgfscope}%
\end{pgfscope}%
\begin{pgfscope}%
\pgfsetbuttcap%
\pgfsetroundjoin%
\definecolor{currentfill}{rgb}{0.000000,0.000000,0.000000}%
\pgfsetfillcolor{currentfill}%
\pgfsetlinewidth{0.602250pt}%
\definecolor{currentstroke}{rgb}{0.000000,0.000000,0.000000}%
\pgfsetstrokecolor{currentstroke}%
\pgfsetdash{}{0pt}%
\pgfsys@defobject{currentmarker}{\pgfqpoint{-0.027778in}{0.000000in}}{\pgfqpoint{0.000000in}{0.000000in}}{%
\pgfpathmoveto{\pgfqpoint{0.000000in}{0.000000in}}%
\pgfpathlineto{\pgfqpoint{-0.027778in}{0.000000in}}%
\pgfusepath{stroke,fill}%
}%
\begin{pgfscope}%
\pgfsys@transformshift{0.418102in}{0.666042in}%
\pgfsys@useobject{currentmarker}{}%
\end{pgfscope}%
\end{pgfscope}%
\begin{pgfscope}%
\pgfsetbuttcap%
\pgfsetroundjoin%
\definecolor{currentfill}{rgb}{0.000000,0.000000,0.000000}%
\pgfsetfillcolor{currentfill}%
\pgfsetlinewidth{0.602250pt}%
\definecolor{currentstroke}{rgb}{0.000000,0.000000,0.000000}%
\pgfsetstrokecolor{currentstroke}%
\pgfsetdash{}{0pt}%
\pgfsys@defobject{currentmarker}{\pgfqpoint{-0.027778in}{0.000000in}}{\pgfqpoint{0.000000in}{0.000000in}}{%
\pgfpathmoveto{\pgfqpoint{0.000000in}{0.000000in}}%
\pgfpathlineto{\pgfqpoint{-0.027778in}{0.000000in}}%
\pgfusepath{stroke,fill}%
}%
\begin{pgfscope}%
\pgfsys@transformshift{0.418102in}{0.777511in}%
\pgfsys@useobject{currentmarker}{}%
\end{pgfscope}%
\end{pgfscope}%
\begin{pgfscope}%
\pgfsetbuttcap%
\pgfsetroundjoin%
\definecolor{currentfill}{rgb}{0.000000,0.000000,0.000000}%
\pgfsetfillcolor{currentfill}%
\pgfsetlinewidth{0.602250pt}%
\definecolor{currentstroke}{rgb}{0.000000,0.000000,0.000000}%
\pgfsetstrokecolor{currentstroke}%
\pgfsetdash{}{0pt}%
\pgfsys@defobject{currentmarker}{\pgfqpoint{-0.027778in}{0.000000in}}{\pgfqpoint{0.000000in}{0.000000in}}{%
\pgfpathmoveto{\pgfqpoint{0.000000in}{0.000000in}}%
\pgfpathlineto{\pgfqpoint{-0.027778in}{0.000000in}}%
\pgfusepath{stroke,fill}%
}%
\begin{pgfscope}%
\pgfsys@transformshift{0.418102in}{1.000449in}%
\pgfsys@useobject{currentmarker}{}%
\end{pgfscope}%
\end{pgfscope}%
\begin{pgfscope}%
\pgfsetbuttcap%
\pgfsetroundjoin%
\definecolor{currentfill}{rgb}{0.000000,0.000000,0.000000}%
\pgfsetfillcolor{currentfill}%
\pgfsetlinewidth{0.602250pt}%
\definecolor{currentstroke}{rgb}{0.000000,0.000000,0.000000}%
\pgfsetstrokecolor{currentstroke}%
\pgfsetdash{}{0pt}%
\pgfsys@defobject{currentmarker}{\pgfqpoint{-0.027778in}{0.000000in}}{\pgfqpoint{0.000000in}{0.000000in}}{%
\pgfpathmoveto{\pgfqpoint{0.000000in}{0.000000in}}%
\pgfpathlineto{\pgfqpoint{-0.027778in}{0.000000in}}%
\pgfusepath{stroke,fill}%
}%
\begin{pgfscope}%
\pgfsys@transformshift{0.418102in}{1.111918in}%
\pgfsys@useobject{currentmarker}{}%
\end{pgfscope}%
\end{pgfscope}%
\begin{pgfscope}%
\pgfsetbuttcap%
\pgfsetroundjoin%
\definecolor{currentfill}{rgb}{0.000000,0.000000,0.000000}%
\pgfsetfillcolor{currentfill}%
\pgfsetlinewidth{0.602250pt}%
\definecolor{currentstroke}{rgb}{0.000000,0.000000,0.000000}%
\pgfsetstrokecolor{currentstroke}%
\pgfsetdash{}{0pt}%
\pgfsys@defobject{currentmarker}{\pgfqpoint{-0.027778in}{0.000000in}}{\pgfqpoint{0.000000in}{0.000000in}}{%
\pgfpathmoveto{\pgfqpoint{0.000000in}{0.000000in}}%
\pgfpathlineto{\pgfqpoint{-0.027778in}{0.000000in}}%
\pgfusepath{stroke,fill}%
}%
\begin{pgfscope}%
\pgfsys@transformshift{0.418102in}{1.223386in}%
\pgfsys@useobject{currentmarker}{}%
\end{pgfscope}%
\end{pgfscope}%
\begin{pgfscope}%
\pgfsetbuttcap%
\pgfsetroundjoin%
\definecolor{currentfill}{rgb}{0.000000,0.000000,0.000000}%
\pgfsetfillcolor{currentfill}%
\pgfsetlinewidth{0.602250pt}%
\definecolor{currentstroke}{rgb}{0.000000,0.000000,0.000000}%
\pgfsetstrokecolor{currentstroke}%
\pgfsetdash{}{0pt}%
\pgfsys@defobject{currentmarker}{\pgfqpoint{-0.027778in}{0.000000in}}{\pgfqpoint{0.000000in}{0.000000in}}{%
\pgfpathmoveto{\pgfqpoint{0.000000in}{0.000000in}}%
\pgfpathlineto{\pgfqpoint{-0.027778in}{0.000000in}}%
\pgfusepath{stroke,fill}%
}%
\begin{pgfscope}%
\pgfsys@transformshift{0.418102in}{1.334855in}%
\pgfsys@useobject{currentmarker}{}%
\end{pgfscope}%
\end{pgfscope}%
\begin{pgfscope}%
\pgfsetbuttcap%
\pgfsetroundjoin%
\definecolor{currentfill}{rgb}{0.000000,0.000000,0.000000}%
\pgfsetfillcolor{currentfill}%
\pgfsetlinewidth{0.602250pt}%
\definecolor{currentstroke}{rgb}{0.000000,0.000000,0.000000}%
\pgfsetstrokecolor{currentstroke}%
\pgfsetdash{}{0pt}%
\pgfsys@defobject{currentmarker}{\pgfqpoint{-0.027778in}{0.000000in}}{\pgfqpoint{0.000000in}{0.000000in}}{%
\pgfpathmoveto{\pgfqpoint{0.000000in}{0.000000in}}%
\pgfpathlineto{\pgfqpoint{-0.027778in}{0.000000in}}%
\pgfusepath{stroke,fill}%
}%
\begin{pgfscope}%
\pgfsys@transformshift{0.418102in}{1.557793in}%
\pgfsys@useobject{currentmarker}{}%
\end{pgfscope}%
\end{pgfscope}%
\begin{pgfscope}%
\pgfsetbuttcap%
\pgfsetroundjoin%
\definecolor{currentfill}{rgb}{0.000000,0.000000,0.000000}%
\pgfsetfillcolor{currentfill}%
\pgfsetlinewidth{0.602250pt}%
\definecolor{currentstroke}{rgb}{0.000000,0.000000,0.000000}%
\pgfsetstrokecolor{currentstroke}%
\pgfsetdash{}{0pt}%
\pgfsys@defobject{currentmarker}{\pgfqpoint{-0.027778in}{0.000000in}}{\pgfqpoint{0.000000in}{0.000000in}}{%
\pgfpathmoveto{\pgfqpoint{0.000000in}{0.000000in}}%
\pgfpathlineto{\pgfqpoint{-0.027778in}{0.000000in}}%
\pgfusepath{stroke,fill}%
}%
\begin{pgfscope}%
\pgfsys@transformshift{0.418102in}{1.669262in}%
\pgfsys@useobject{currentmarker}{}%
\end{pgfscope}%
\end{pgfscope}%
\begin{pgfscope}%
\pgfsetbuttcap%
\pgfsetroundjoin%
\definecolor{currentfill}{rgb}{0.000000,0.000000,0.000000}%
\pgfsetfillcolor{currentfill}%
\pgfsetlinewidth{0.602250pt}%
\definecolor{currentstroke}{rgb}{0.000000,0.000000,0.000000}%
\pgfsetstrokecolor{currentstroke}%
\pgfsetdash{}{0pt}%
\pgfsys@defobject{currentmarker}{\pgfqpoint{-0.027778in}{0.000000in}}{\pgfqpoint{0.000000in}{0.000000in}}{%
\pgfpathmoveto{\pgfqpoint{0.000000in}{0.000000in}}%
\pgfpathlineto{\pgfqpoint{-0.027778in}{0.000000in}}%
\pgfusepath{stroke,fill}%
}%
\begin{pgfscope}%
\pgfsys@transformshift{0.418102in}{1.780731in}%
\pgfsys@useobject{currentmarker}{}%
\end{pgfscope}%
\end{pgfscope}%
\begin{pgfscope}%
\pgfsetbuttcap%
\pgfsetroundjoin%
\definecolor{currentfill}{rgb}{0.000000,0.000000,0.000000}%
\pgfsetfillcolor{currentfill}%
\pgfsetlinewidth{0.602250pt}%
\definecolor{currentstroke}{rgb}{0.000000,0.000000,0.000000}%
\pgfsetstrokecolor{currentstroke}%
\pgfsetdash{}{0pt}%
\pgfsys@defobject{currentmarker}{\pgfqpoint{-0.027778in}{0.000000in}}{\pgfqpoint{0.000000in}{0.000000in}}{%
\pgfpathmoveto{\pgfqpoint{0.000000in}{0.000000in}}%
\pgfpathlineto{\pgfqpoint{-0.027778in}{0.000000in}}%
\pgfusepath{stroke,fill}%
}%
\begin{pgfscope}%
\pgfsys@transformshift{0.418102in}{1.892200in}%
\pgfsys@useobject{currentmarker}{}%
\end{pgfscope}%
\end{pgfscope}%
\begin{pgfscope}%
\pgfsetbuttcap%
\pgfsetroundjoin%
\definecolor{currentfill}{rgb}{0.000000,0.000000,0.000000}%
\pgfsetfillcolor{currentfill}%
\pgfsetlinewidth{0.602250pt}%
\definecolor{currentstroke}{rgb}{0.000000,0.000000,0.000000}%
\pgfsetstrokecolor{currentstroke}%
\pgfsetdash{}{0pt}%
\pgfsys@defobject{currentmarker}{\pgfqpoint{-0.027778in}{0.000000in}}{\pgfqpoint{0.000000in}{0.000000in}}{%
\pgfpathmoveto{\pgfqpoint{0.000000in}{0.000000in}}%
\pgfpathlineto{\pgfqpoint{-0.027778in}{0.000000in}}%
\pgfusepath{stroke,fill}%
}%
\begin{pgfscope}%
\pgfsys@transformshift{0.418102in}{2.115138in}%
\pgfsys@useobject{currentmarker}{}%
\end{pgfscope}%
\end{pgfscope}%
\begin{pgfscope}%
\pgfsetbuttcap%
\pgfsetroundjoin%
\definecolor{currentfill}{rgb}{0.000000,0.000000,0.000000}%
\pgfsetfillcolor{currentfill}%
\pgfsetlinewidth{0.602250pt}%
\definecolor{currentstroke}{rgb}{0.000000,0.000000,0.000000}%
\pgfsetstrokecolor{currentstroke}%
\pgfsetdash{}{0pt}%
\pgfsys@defobject{currentmarker}{\pgfqpoint{-0.027778in}{0.000000in}}{\pgfqpoint{0.000000in}{0.000000in}}{%
\pgfpathmoveto{\pgfqpoint{0.000000in}{0.000000in}}%
\pgfpathlineto{\pgfqpoint{-0.027778in}{0.000000in}}%
\pgfusepath{stroke,fill}%
}%
\begin{pgfscope}%
\pgfsys@transformshift{0.418102in}{2.226607in}%
\pgfsys@useobject{currentmarker}{}%
\end{pgfscope}%
\end{pgfscope}%
\begin{pgfscope}%
\pgfsetbuttcap%
\pgfsetroundjoin%
\definecolor{currentfill}{rgb}{0.000000,0.000000,0.000000}%
\pgfsetfillcolor{currentfill}%
\pgfsetlinewidth{0.602250pt}%
\definecolor{currentstroke}{rgb}{0.000000,0.000000,0.000000}%
\pgfsetstrokecolor{currentstroke}%
\pgfsetdash{}{0pt}%
\pgfsys@defobject{currentmarker}{\pgfqpoint{-0.027778in}{0.000000in}}{\pgfqpoint{0.000000in}{0.000000in}}{%
\pgfpathmoveto{\pgfqpoint{0.000000in}{0.000000in}}%
\pgfpathlineto{\pgfqpoint{-0.027778in}{0.000000in}}%
\pgfusepath{stroke,fill}%
}%
\begin{pgfscope}%
\pgfsys@transformshift{0.418102in}{2.338076in}%
\pgfsys@useobject{currentmarker}{}%
\end{pgfscope}%
\end{pgfscope}%
\begin{pgfscope}%
\pgfsetbuttcap%
\pgfsetroundjoin%
\definecolor{currentfill}{rgb}{0.000000,0.000000,0.000000}%
\pgfsetfillcolor{currentfill}%
\pgfsetlinewidth{0.602250pt}%
\definecolor{currentstroke}{rgb}{0.000000,0.000000,0.000000}%
\pgfsetstrokecolor{currentstroke}%
\pgfsetdash{}{0pt}%
\pgfsys@defobject{currentmarker}{\pgfqpoint{-0.027778in}{0.000000in}}{\pgfqpoint{0.000000in}{0.000000in}}{%
\pgfpathmoveto{\pgfqpoint{0.000000in}{0.000000in}}%
\pgfpathlineto{\pgfqpoint{-0.027778in}{0.000000in}}%
\pgfusepath{stroke,fill}%
}%
\begin{pgfscope}%
\pgfsys@transformshift{0.418102in}{2.449545in}%
\pgfsys@useobject{currentmarker}{}%
\end{pgfscope}%
\end{pgfscope}%
\begin{pgfscope}%
\pgfsetbuttcap%
\pgfsetroundjoin%
\definecolor{currentfill}{rgb}{0.000000,0.000000,0.000000}%
\pgfsetfillcolor{currentfill}%
\pgfsetlinewidth{0.602250pt}%
\definecolor{currentstroke}{rgb}{0.000000,0.000000,0.000000}%
\pgfsetstrokecolor{currentstroke}%
\pgfsetdash{}{0pt}%
\pgfsys@defobject{currentmarker}{\pgfqpoint{-0.027778in}{0.000000in}}{\pgfqpoint{0.000000in}{0.000000in}}{%
\pgfpathmoveto{\pgfqpoint{0.000000in}{0.000000in}}%
\pgfpathlineto{\pgfqpoint{-0.027778in}{0.000000in}}%
\pgfusepath{stroke,fill}%
}%
\begin{pgfscope}%
\pgfsys@transformshift{0.418102in}{2.672483in}%
\pgfsys@useobject{currentmarker}{}%
\end{pgfscope}%
\end{pgfscope}%
\begin{pgfscope}%
\pgfsetbuttcap%
\pgfsetroundjoin%
\definecolor{currentfill}{rgb}{0.000000,0.000000,0.000000}%
\pgfsetfillcolor{currentfill}%
\pgfsetlinewidth{0.602250pt}%
\definecolor{currentstroke}{rgb}{0.000000,0.000000,0.000000}%
\pgfsetstrokecolor{currentstroke}%
\pgfsetdash{}{0pt}%
\pgfsys@defobject{currentmarker}{\pgfqpoint{-0.027778in}{0.000000in}}{\pgfqpoint{0.000000in}{0.000000in}}{%
\pgfpathmoveto{\pgfqpoint{0.000000in}{0.000000in}}%
\pgfpathlineto{\pgfqpoint{-0.027778in}{0.000000in}}%
\pgfusepath{stroke,fill}%
}%
\begin{pgfscope}%
\pgfsys@transformshift{0.418102in}{2.783952in}%
\pgfsys@useobject{currentmarker}{}%
\end{pgfscope}%
\end{pgfscope}%
\begin{pgfscope}%
\pgfsetbuttcap%
\pgfsetroundjoin%
\definecolor{currentfill}{rgb}{0.000000,0.000000,0.000000}%
\pgfsetfillcolor{currentfill}%
\pgfsetlinewidth{0.602250pt}%
\definecolor{currentstroke}{rgb}{0.000000,0.000000,0.000000}%
\pgfsetstrokecolor{currentstroke}%
\pgfsetdash{}{0pt}%
\pgfsys@defobject{currentmarker}{\pgfqpoint{-0.027778in}{0.000000in}}{\pgfqpoint{0.000000in}{0.000000in}}{%
\pgfpathmoveto{\pgfqpoint{0.000000in}{0.000000in}}%
\pgfpathlineto{\pgfqpoint{-0.027778in}{0.000000in}}%
\pgfusepath{stroke,fill}%
}%
\begin{pgfscope}%
\pgfsys@transformshift{0.418102in}{2.895421in}%
\pgfsys@useobject{currentmarker}{}%
\end{pgfscope}%
\end{pgfscope}%
\begin{pgfscope}%
\pgfsetbuttcap%
\pgfsetroundjoin%
\definecolor{currentfill}{rgb}{0.000000,0.000000,0.000000}%
\pgfsetfillcolor{currentfill}%
\pgfsetlinewidth{0.602250pt}%
\definecolor{currentstroke}{rgb}{0.000000,0.000000,0.000000}%
\pgfsetstrokecolor{currentstroke}%
\pgfsetdash{}{0pt}%
\pgfsys@defobject{currentmarker}{\pgfqpoint{-0.027778in}{0.000000in}}{\pgfqpoint{0.000000in}{0.000000in}}{%
\pgfpathmoveto{\pgfqpoint{0.000000in}{0.000000in}}%
\pgfpathlineto{\pgfqpoint{-0.027778in}{0.000000in}}%
\pgfusepath{stroke,fill}%
}%
\begin{pgfscope}%
\pgfsys@transformshift{0.418102in}{3.006889in}%
\pgfsys@useobject{currentmarker}{}%
\end{pgfscope}%
\end{pgfscope}%
\begin{pgfscope}%
\pgfsetbuttcap%
\pgfsetroundjoin%
\definecolor{currentfill}{rgb}{0.000000,0.000000,0.000000}%
\pgfsetfillcolor{currentfill}%
\pgfsetlinewidth{0.602250pt}%
\definecolor{currentstroke}{rgb}{0.000000,0.000000,0.000000}%
\pgfsetstrokecolor{currentstroke}%
\pgfsetdash{}{0pt}%
\pgfsys@defobject{currentmarker}{\pgfqpoint{-0.027778in}{0.000000in}}{\pgfqpoint{0.000000in}{0.000000in}}{%
\pgfpathmoveto{\pgfqpoint{0.000000in}{0.000000in}}%
\pgfpathlineto{\pgfqpoint{-0.027778in}{0.000000in}}%
\pgfusepath{stroke,fill}%
}%
\begin{pgfscope}%
\pgfsys@transformshift{0.418102in}{3.229827in}%
\pgfsys@useobject{currentmarker}{}%
\end{pgfscope}%
\end{pgfscope}%
\begin{pgfscope}%
\pgfsetbuttcap%
\pgfsetroundjoin%
\definecolor{currentfill}{rgb}{0.000000,0.000000,0.000000}%
\pgfsetfillcolor{currentfill}%
\pgfsetlinewidth{0.602250pt}%
\definecolor{currentstroke}{rgb}{0.000000,0.000000,0.000000}%
\pgfsetstrokecolor{currentstroke}%
\pgfsetdash{}{0pt}%
\pgfsys@defobject{currentmarker}{\pgfqpoint{-0.027778in}{0.000000in}}{\pgfqpoint{0.000000in}{0.000000in}}{%
\pgfpathmoveto{\pgfqpoint{0.000000in}{0.000000in}}%
\pgfpathlineto{\pgfqpoint{-0.027778in}{0.000000in}}%
\pgfusepath{stroke,fill}%
}%
\begin{pgfscope}%
\pgfsys@transformshift{0.418102in}{3.341296in}%
\pgfsys@useobject{currentmarker}{}%
\end{pgfscope}%
\end{pgfscope}%
\begin{pgfscope}%
\pgfsetbuttcap%
\pgfsetroundjoin%
\definecolor{currentfill}{rgb}{0.000000,0.000000,0.000000}%
\pgfsetfillcolor{currentfill}%
\pgfsetlinewidth{0.602250pt}%
\definecolor{currentstroke}{rgb}{0.000000,0.000000,0.000000}%
\pgfsetstrokecolor{currentstroke}%
\pgfsetdash{}{0pt}%
\pgfsys@defobject{currentmarker}{\pgfqpoint{-0.027778in}{0.000000in}}{\pgfqpoint{0.000000in}{0.000000in}}{%
\pgfpathmoveto{\pgfqpoint{0.000000in}{0.000000in}}%
\pgfpathlineto{\pgfqpoint{-0.027778in}{0.000000in}}%
\pgfusepath{stroke,fill}%
}%
\begin{pgfscope}%
\pgfsys@transformshift{0.418102in}{3.452765in}%
\pgfsys@useobject{currentmarker}{}%
\end{pgfscope}%
\end{pgfscope}%
\begin{pgfscope}%
\pgfsetbuttcap%
\pgfsetroundjoin%
\definecolor{currentfill}{rgb}{0.000000,0.000000,0.000000}%
\pgfsetfillcolor{currentfill}%
\pgfsetlinewidth{0.602250pt}%
\definecolor{currentstroke}{rgb}{0.000000,0.000000,0.000000}%
\pgfsetstrokecolor{currentstroke}%
\pgfsetdash{}{0pt}%
\pgfsys@defobject{currentmarker}{\pgfqpoint{-0.027778in}{0.000000in}}{\pgfqpoint{0.000000in}{0.000000in}}{%
\pgfpathmoveto{\pgfqpoint{0.000000in}{0.000000in}}%
\pgfpathlineto{\pgfqpoint{-0.027778in}{0.000000in}}%
\pgfusepath{stroke,fill}%
}%
\begin{pgfscope}%
\pgfsys@transformshift{0.418102in}{3.564234in}%
\pgfsys@useobject{currentmarker}{}%
\end{pgfscope}%
\end{pgfscope}%
\begin{pgfscope}%
\pgfpathrectangle{\pgfqpoint{0.313819in}{0.264754in}}{\pgfqpoint{5.422710in}{3.477831in}}%
\pgfusepath{clip}%
\pgfsetroundcap%
\pgfsetroundjoin%
\pgfsetlinewidth{2.007500pt}%
\definecolor{currentstroke}{rgb}{0.000000,0.000000,1.000000}%
\pgfsetstrokecolor{currentstroke}%
\pgfsetdash{}{0pt}%
\pgfpathmoveto{\pgfqpoint{0.418102in}{0.331635in}}%
\pgfpathquadraticcurveto{\pgfqpoint{1.287126in}{0.888980in}}{\pgfqpoint{2.130008in}{1.429558in}}%
\pgfusepath{stroke}%
\end{pgfscope}%
\begin{pgfscope}%
\pgfpathrectangle{\pgfqpoint{0.313819in}{0.264754in}}{\pgfqpoint{5.422710in}{3.477831in}}%
\pgfusepath{clip}%
\pgfsetroundcap%
\pgfsetroundjoin%
\definecolor{currentfill}{rgb}{0.000000,0.000000,1.000000}%
\pgfsetfillcolor{currentfill}%
\pgfsetlinewidth{2.007500pt}%
\definecolor{currentstroke}{rgb}{0.000000,0.000000,1.000000}%
\pgfsetstrokecolor{currentstroke}%
\pgfsetdash{}{0pt}%
\pgfpathmoveto{\pgfqpoint{1.975607in}{1.413034in}}%
\pgfpathlineto{\pgfqpoint{2.130008in}{1.429558in}}%
\pgfpathlineto{\pgfqpoint{2.050587in}{1.296123in}}%
\pgfpathlineto{\pgfqpoint{1.975607in}{1.413034in}}%
\pgfpathclose%
\pgfusepath{stroke,fill}%
\end{pgfscope}%
\begin{pgfscope}%
\pgfpathrectangle{\pgfqpoint{0.313819in}{0.264754in}}{\pgfqpoint{5.422710in}{3.477831in}}%
\pgfusepath{clip}%
\pgfsetroundcap%
\pgfsetroundjoin%
\pgfsetlinewidth{2.007500pt}%
\definecolor{currentstroke}{rgb}{0.000000,0.000000,1.000000}%
\pgfsetstrokecolor{currentstroke}%
\pgfsetdash{}{0pt}%
\pgfpathmoveto{\pgfqpoint{0.418102in}{0.331635in}}%
\pgfpathquadraticcurveto{\pgfqpoint{1.287126in}{2.003669in}}{\pgfqpoint{2.141827in}{3.648146in}}%
\pgfusepath{stroke}%
\end{pgfscope}%
\begin{pgfscope}%
\pgfpathrectangle{\pgfqpoint{0.313819in}{0.264754in}}{\pgfqpoint{5.422710in}{3.477831in}}%
\pgfusepath{clip}%
\pgfsetroundcap%
\pgfsetroundjoin%
\definecolor{currentfill}{rgb}{0.000000,0.000000,1.000000}%
\pgfsetfillcolor{currentfill}%
\pgfsetlinewidth{2.007500pt}%
\definecolor{currentstroke}{rgb}{0.000000,0.000000,1.000000}%
\pgfsetstrokecolor{currentstroke}%
\pgfsetdash{}{0pt}%
\pgfpathmoveto{\pgfqpoint{2.016157in}{3.556934in}}%
\pgfpathlineto{\pgfqpoint{2.141827in}{3.648146in}}%
\pgfpathlineto{\pgfqpoint{2.139395in}{3.492883in}}%
\pgfpathlineto{\pgfqpoint{2.016157in}{3.556934in}}%
\pgfpathclose%
\pgfusepath{stroke,fill}%
\end{pgfscope}%
\begin{pgfscope}%
\pgfpathrectangle{\pgfqpoint{0.313819in}{0.264754in}}{\pgfqpoint{5.422710in}{3.477831in}}%
\pgfusepath{clip}%
\pgfsetroundcap%
\pgfsetroundjoin%
\pgfsetlinewidth{2.007500pt}%
\definecolor{currentstroke}{rgb}{0.000000,0.000000,1.000000}%
\pgfsetstrokecolor{currentstroke}%
\pgfsetdash{}{0pt}%
\pgfpathmoveto{\pgfqpoint{0.418102in}{0.331635in}}%
\pgfpathquadraticcurveto{\pgfqpoint{2.156150in}{1.446324in}}{\pgfqpoint{3.868056in}{2.544248in}}%
\pgfusepath{stroke}%
\end{pgfscope}%
\begin{pgfscope}%
\pgfpathrectangle{\pgfqpoint{0.313819in}{0.264754in}}{\pgfqpoint{5.422710in}{3.477831in}}%
\pgfusepath{clip}%
\pgfsetroundcap%
\pgfsetroundjoin%
\definecolor{currentfill}{rgb}{0.000000,0.000000,1.000000}%
\pgfsetfillcolor{currentfill}%
\pgfsetlinewidth{2.007500pt}%
\definecolor{currentstroke}{rgb}{0.000000,0.000000,1.000000}%
\pgfsetstrokecolor{currentstroke}%
\pgfsetdash{}{0pt}%
\pgfpathmoveto{\pgfqpoint{3.713655in}{2.527723in}}%
\pgfpathlineto{\pgfqpoint{3.868056in}{2.544248in}}%
\pgfpathlineto{\pgfqpoint{3.788635in}{2.410812in}}%
\pgfpathlineto{\pgfqpoint{3.713655in}{2.527723in}}%
\pgfpathclose%
\pgfusepath{stroke,fill}%
\end{pgfscope}%
\begin{pgfscope}%
\pgfpathrectangle{\pgfqpoint{0.313819in}{0.264754in}}{\pgfqpoint{5.422710in}{3.477831in}}%
\pgfusepath{clip}%
\pgfsetroundcap%
\pgfsetroundjoin%
\pgfsetlinewidth{2.007500pt}%
\definecolor{currentstroke}{rgb}{0.000000,0.501961,0.000000}%
\pgfsetstrokecolor{currentstroke}%
\pgfsetdash{}{0pt}%
\pgfpathmoveto{\pgfqpoint{0.418102in}{0.331635in}}%
\pgfpathquadraticcurveto{\pgfqpoint{3.025174in}{0.888980in}}{\pgfqpoint{5.601876in}{1.439832in}}%
\pgfusepath{stroke}%
\end{pgfscope}%
\begin{pgfscope}%
\pgfpathrectangle{\pgfqpoint{0.313819in}{0.264754in}}{\pgfqpoint{5.422710in}{3.477831in}}%
\pgfusepath{clip}%
\pgfsetroundcap%
\pgfsetroundjoin%
\definecolor{currentfill}{rgb}{0.000000,0.501961,0.000000}%
\pgfsetfillcolor{currentfill}%
\pgfsetlinewidth{2.007500pt}%
\definecolor{currentstroke}{rgb}{0.000000,0.501961,0.000000}%
\pgfsetstrokecolor{currentstroke}%
\pgfsetdash{}{0pt}%
\pgfpathmoveto{\pgfqpoint{5.451538in}{1.478706in}}%
\pgfpathlineto{\pgfqpoint{5.601876in}{1.439832in}}%
\pgfpathlineto{\pgfqpoint{5.480574in}{1.342886in}}%
\pgfpathlineto{\pgfqpoint{5.451538in}{1.478706in}}%
\pgfpathclose%
\pgfusepath{stroke,fill}%
\end{pgfscope}%
\begin{pgfscope}%
\pgfpathrectangle{\pgfqpoint{0.313819in}{0.264754in}}{\pgfqpoint{5.422710in}{3.477831in}}%
\pgfusepath{clip}%
\pgfsetroundcap%
\pgfsetroundjoin%
\pgfsetlinewidth{2.007500pt}%
\definecolor{currentstroke}{rgb}{1.000000,0.000000,0.000000}%
\pgfsetstrokecolor{currentstroke}%
\pgfsetdash{}{0pt}%
\pgfpathmoveto{\pgfqpoint{3.894198in}{2.561014in}}%
\pgfpathquadraticcurveto{\pgfqpoint{3.025174in}{3.118358in}}{\pgfqpoint{2.182292in}{3.658937in}}%
\pgfusepath{stroke}%
\end{pgfscope}%
\begin{pgfscope}%
\pgfpathrectangle{\pgfqpoint{0.313819in}{0.264754in}}{\pgfqpoint{5.422710in}{3.477831in}}%
\pgfusepath{clip}%
\pgfsetroundcap%
\pgfsetroundjoin%
\definecolor{currentfill}{rgb}{1.000000,0.000000,0.000000}%
\pgfsetfillcolor{currentfill}%
\pgfsetlinewidth{2.007500pt}%
\definecolor{currentstroke}{rgb}{1.000000,0.000000,0.000000}%
\pgfsetstrokecolor{currentstroke}%
\pgfsetdash{}{0pt}%
\pgfpathmoveto{\pgfqpoint{2.261712in}{3.525502in}}%
\pgfpathlineto{\pgfqpoint{2.182292in}{3.658937in}}%
\pgfpathlineto{\pgfqpoint{2.336693in}{3.642412in}}%
\pgfpathlineto{\pgfqpoint{2.261712in}{3.525502in}}%
\pgfpathclose%
\pgfusepath{stroke,fill}%
\end{pgfscope}%
\begin{pgfscope}%
\pgfpathrectangle{\pgfqpoint{0.313819in}{0.264754in}}{\pgfqpoint{5.422710in}{3.477831in}}%
\pgfusepath{clip}%
\pgfsetroundcap%
\pgfsetroundjoin%
\pgfsetlinewidth{2.007500pt}%
\definecolor{currentstroke}{rgb}{1.000000,0.000000,0.000000}%
\pgfsetstrokecolor{currentstroke}%
\pgfsetdash{}{0pt}%
\pgfpathmoveto{\pgfqpoint{3.894198in}{2.561014in}}%
\pgfpathquadraticcurveto{\pgfqpoint{4.763222in}{2.003669in}}{\pgfqpoint{5.606104in}{1.463090in}}%
\pgfusepath{stroke}%
\end{pgfscope}%
\begin{pgfscope}%
\pgfpathrectangle{\pgfqpoint{0.313819in}{0.264754in}}{\pgfqpoint{5.422710in}{3.477831in}}%
\pgfusepath{clip}%
\pgfsetroundcap%
\pgfsetroundjoin%
\definecolor{currentfill}{rgb}{1.000000,0.000000,0.000000}%
\pgfsetfillcolor{currentfill}%
\pgfsetlinewidth{2.007500pt}%
\definecolor{currentstroke}{rgb}{1.000000,0.000000,0.000000}%
\pgfsetstrokecolor{currentstroke}%
\pgfsetdash{}{0pt}%
\pgfpathmoveto{\pgfqpoint{5.526684in}{1.596526in}}%
\pgfpathlineto{\pgfqpoint{5.606104in}{1.463090in}}%
\pgfpathlineto{\pgfqpoint{5.451703in}{1.479615in}}%
\pgfpathlineto{\pgfqpoint{5.526684in}{1.596526in}}%
\pgfpathclose%
\pgfusepath{stroke,fill}%
\end{pgfscope}%
\begin{pgfscope}%
\pgfsetrectcap%
\pgfsetmiterjoin%
\pgfsetlinewidth{0.803000pt}%
\definecolor{currentstroke}{rgb}{0.000000,0.000000,0.000000}%
\pgfsetstrokecolor{currentstroke}%
\pgfsetdash{}{0pt}%
\pgfpathmoveto{\pgfqpoint{0.418102in}{0.264754in}}%
\pgfpathlineto{\pgfqpoint{0.418102in}{3.742584in}}%
\pgfusepath{stroke}%
\end{pgfscope}%
\begin{pgfscope}%
\pgfsetrectcap%
\pgfsetmiterjoin%
\pgfsetlinewidth{0.803000pt}%
\definecolor{currentstroke}{rgb}{0.000000,0.000000,0.000000}%
\pgfsetstrokecolor{currentstroke}%
\pgfsetdash{}{0pt}%
\pgfpathmoveto{\pgfqpoint{0.313819in}{0.331635in}}%
\pgfpathlineto{\pgfqpoint{5.736529in}{0.331635in}}%
\pgfusepath{stroke}%
\end{pgfscope}%
\end{pgfpicture}%
\makeatother%
\endgroup%

	\section{Projektionsmatricen}
	Det er keometrisk klart at $proj_U, comp_U, refl_U : \RR^n \rightarrow \RR^n$ er lineære
	transformationer. Derfor findes $nxn $ matricer er 
	\begin{itemize}
		\item P - Projektionsmatricen for $U$
		\item C - Komponentmatricen for $U$
		\item R - Spejlingsmatricen for $U$
	\end{itemize}
	Projektionsmatricerne\\
	\[ P = \frac{uu^T}{u^Tu} \]\emph{Fordi $uu^T$ er en søjlevektor gange rækkevektor får vi en
		matrice. $u^T$ er række gange søjle, og derfor en skalar. Derfor får vi matrice
	divideret med skalar}
	\[ C = I - P\]
	\[ R = 2P - I\]

	\section{Ortonormale baser}
	Et set af vektorer $u_k$ kaldes
	\begin{itemize}
		\item Parvist ortogonale hvis $u_i \bullet u_j = 0$ for alle $i \neq j$
		\item Parvist ortonormale hvis $u_i \bullet u_j = 0$ for alle $i \neq j$ og 
			$||u_i|| = 1$ for alle i
	\end{itemize}

	Ethvert sæt $S$ af parvist ortogonale ikke-nul vektorer i $\RR^n$ er lineært uafhængige\\
	En ortogonal/ortonormal basis for et underrum $U$ af $\RR^n$ er en basis $B$ for $U$ hvori
	vektorerne er parvist ortogonale/ortonormale

	Hvis vi har en ortogonal base $B$ så kan vi udregne koordinater: 
	\[ [v]_B = \begin{pmatrix} \frac{v\bullet u_1}{||u_1||^2}\\\dots\\\frac{v\bullet u_k}{||u_k||^2}\end{pmatrix}\]

	\section{Ortogonale Matrice}
	Vi har en matrix $A$ og søjlerne i $A$ er parvist ortonormale hvis og kun hvis $A^TA = I_K$
	altså hvis $A^T$ er venstre-invers til $A$\\

	En kvatratisk matrix $Q$ kaldes ortogonal hvis søjlerne i $Q$ er parvist ortonormale altså
	hvis $Q^TQ = I_n$. Det gælder for den ortogonale matrice $Q$ gælder $Q^{-1} = Q^T$\\

	Ortogonale matricer bevarer prikprodukt: $Qu \bullet Qv = u \bullet v$.\\

	\subsection{Ortogonale lineære transformationer}
	Hvis vi har en lineær transformation hvor transformationsmatricen er ortogonal så får vi\\
	Længden af den vektor vi starter med, og den transformerede vektor så bevarer de længden.\\
	Typisk: Rotationer og spejlinger

	\chapter{16/05}
	\section{Ortogonal komplement}
	Definition:\\
	To underrum $u$ og $v$ af $\RR^n$ kaldes ortogonale hvis enhver vektor i $u$ er ortogonal
	på enhver vektor i $V$ dvs $u \bullet v = 0 \forall u \in U, v \in V$\\
	
	Thm 4.8:\\
	Lad $U = \text{span}\{u_1,\dots,u_p\}$ og $V = \text{span}\{v_1,\dots,v_q\}$
	For at checke om underrum er ortogonale er det nok at checke `frembringersæt'. Altså hvis
	du har to baser $U = \{u_1, u_2\}, V = \{v_1, v_2, v_3\}$ skal vi bare checke 
	\[ u_1 \bullet v_1 = u_2 \bullet v_1 = u_1 \bullet v_2 = u_2 \bullet v_2 = u_1 \bullet v_3= 
	u_2 \bullet v_3 = 0\]


	Def 4.11\\
	Lad $U$ være et underrum af $\RR^n$ det ortogonale komplement $U^\perp$ til $U$ består af
	samtlige vektorer som er ortogonale på alle vektorer i $U$: Det gælder
	\begin{itemize}
		\item $U^\perp$ er et underrum af $\RR^n$
		\item $U^\perp \cap U = \{0\}$
		\item $(U^\perp)^\perp = U$
	\end{itemize}
	Ortogonal komplementet til en linie er et plan\\

	Thm 4.10. For enhver matrix $A$ Gælder \[ (row A)^\perp = null A\]
	\[(col A)^\perp = null A^\perp\]\[(null A)^\perp = row A\]
	Vi har også $dim U + dim U^\perp = n$ hvor n er fra $\RR^n$

	\section{Ortogonal projektion}
	Vi har lært en-dimensionel projektion(eg vektor til vektor)\\ Vi skal nu projektere generelt

	Den ortogonale projektion af $v$ på $U$ er 
	\[ proj_U(v) = \frac{v\bullet u_1}{||u_1||^2 u_1}+ \dots + 
		\frac{v\bullet u_k}{||u_k||^2 u_k }
		\]
			\[comp_U(v) = v-proj_U(v)\]

	\section{Gram-Schmidt processen}
	Gram-schmidt kan lave et ortogonalbasis til en ortonormalbasis\\
	Lad $u_1, \dots, u_n$ være lineært uafhængige vektorer, og dermed en basis for underrummet
	$U = span u_1 u_n$. Vi normaliserer så
	\[ q_1 = \frac{u_1}{||u_1||}\]
	MANGLER

	\section{QR faktorisering}
	Det er en måde at skrive en matrix som et produkt af to pæne matrixer, vi definerer $u$ som søjler
	i $A$\\
	Man laver Gram-Schmidt, og holder styr på tallene $r_{ij}$
	\[ q_1 = \frac{u_1}{||u_1||} = \frac{u_1}{r_{11}} \]
	\[ q_2 = \frac{u_2 - (u_2 \bullet q_1)q_1}{||u_2 - (u_2 \cdot q_1)q_1} = \frac{u_2 - r_{12}q_1}{r_{22}}\]
	Så har vi $r_{ij}$
	\[ A = QR = (q_1 |q_2 | \dots | q_n)
		\begin{pmatrix}
			r_{11}&r_{12}&\dots&r_{1j}\\
			0 & r_{22} & \dots r_{2j}\\
		\end{pmatrix}
	\]

	\section{Mindste kvadraters metode}
	fFor en $mxn$ matrix $A$ betragter vi et lineært ligningssystem $Ax = b$, hvori der
	er mange ligninger med få ubekendte. Hvis ligningssystemet er inkonsistent, så efterspørger
	vi i stedet den vektor $x$ som gør størrelsen $||b-Ax||$ mindst muligt. Altså den $x$ der
	kommer tættest på $b$\\
	Thm 4.13 viser at vi skal vælge $x = \bar{x}$ hvor $A\bar{x} = proj_{col A}(b)$\\
	For $\bar{x}$ gælder altså at $b-A\bar{x}$ er ortogonal til søjlerummet. Derfor ved vi at
	$A^T(b-A\bar{x}) = 0$ Den søgte vektor $\bar{x}$ tilfredsstiller altså følgende ligningssystem
	$A^TA\bar{x} = A^Tb$
	Vi får så næsten altiD(praktisk altid)
	\[ \bar{x} = (A^TA)^{-1}A^Tb \] Altså den vektor $\bar{x}$ der kommer tættest på at løse
	ligningssystemet\\

	Et eksempel på brug kunne være en samling datapunkter med tilsvarende x,y koordinater, 
	vi har nu mange datapunkter(ligninger $y = ax+b$) og to ubekendte $a,b$)

	\chapter{forlasning 11}
	\section{Diagonalisering af matricer}
	Det drejer sig om ligningen \[P^{-1}AP = \begin{bmatrix}\lambda_1&0&0\\0&\lambda_2&0\\0&0&\lambda_3\end{bmatrix}\]
	En $n x n$ matrix kaldes diagonaliserbar hvis der findes en matrix P og en diagonalmatrix som opfylder
	$P^{-1}AP = D$.
	Matricen $A - (1,1,)(-2,4)$ er diagonaliserbar fordi $P = (1,1)(1,2)$ og $D = (2,0)(0,3)$\\

	\section{Lidt teori}
	Vi antager at $A$ er diagonaliserbar n x n matrix, dcs $P^{-1}AP=D$. Vi kan rykke lidt
	rundt og forgange med P på begge sider $P^{-1}AP = D \Rightarrow AP=PD$ og vi bemærker
	$Av_1 | \dots | Av_n) = AP = PD = (\lambda_1v_1|\dots|\lambda_nv_n)$
	Vi konkluderer at 
	\begin{itemize}
		\item $\lambda$ er en egenværdi for $A$ med tilhørende egenvektor $v$.
		\item $\{v_1, \dots, v_n\}$ er en basis for $\RR^n$ bestående af egenvektorer i $A$
	\end{itemize}
	
	\begin{theorem}{6.4}
		Lad A være en n x n matrix med k indbyrdes forskellige komplekse egenværdier.
		Vælg for hver $1 \leq i \leq k$ en basis $B_i$ for egenrumet $E_{\lambda_i} = 
		Null(A - \lambda_iI)$ så er mængden $B = B_1 \cup \dots \cup B_k$ lineært uafhængig
		og der gækder: $A$ er diagonaliserbar og antallet af vektorer i B = n
	\end{theorem}

	\section{En metode}
	Lad $A$ være en n x n matrix med indbyrdes forskellige komplekse egenværdier $\lambda_1,
	\dots, \lambda_k$ hvor $k \leq n$\\
	\begin{itemize}
		\item Vælg basis $B_1 = v_1, \dots, v_{g_{\lambda_1}}$ for $E_{\lambda_1} = Null(A-\lambda_{1} \cdot I) $
		\item $\dots$
		\item Vælg basis $B_k = v_1, \dots, v_{g_{\lambda_k}}$ for $E_{\lambda_k} = Null(A-\lambda_k \cdot I)$
	\end{itemize}
	Vi opstiller så de baser som søjler i en
	

	\section{Øvelse}
	Vi får en matrix $A$
	\[ A = \begin{pmatrix}2 & 3\\3 & 2\end{pmatrix} \]
	Og vil bestemme $P$ og $D$. Vi bruger sage
	\begin{verbatim}

sage: a = matrix(SR,2,2,[2,3,3,2])
sage: a.eigenvalues()
[5, -1]
sage: a - (5*identity_matrix(2))
[-3  3]
[ 3 -3]
sage: (a - (5*identity_matrix(2))).rref()
[ 1 -1]
[ 0  0]
sage: b1 = vector([1,1])
sage: (a + identity_matrix(2))
[3 3]
[3 3]
sage: (a + identity_matrix(2)).rref()
[1 1]
[0 0]
sage: b2 = vector([-1,1])
sage: P = matrix(SR,2,2,[b1,b2]).T
sage: P
[ 1 -1]
[ 1  1]
sage: P.inverse()*a*P
[ 5  0]
[ 0 -1]

\end{verbatim}
	Og så har vi fundet P og D
	\chapter{04/06}
	\section{Markovkæder}
	En markovkæde er et dynamisk system der opstår ud fra tilstande og 
	overgangssandsyndligheder. \\
	Betragt et system: En vilkårlig datalog\\
	Tilstange:\\
	\begin{itemize}
		\item[O1]De arbejder i firma A
		\item[O2]De arbejder i firma B
		\item[O3]De arbejder i firma C
	\end{itemize}
	Du kan opstille sandsynligheder:
	\begin{tabular}{c|ccc}
		Til\Fra&O1&O2&O3\\
		O1&0.6&0.2&0.4\\
		O2&0.2&0.6&0.4\\
		O3&0.2&0.2&0.2\\
	\end{tabular}
	Så hvis datalogen i et givent år arbejder i firma A, så er der 20\% chance for at han
	året efter tager et arbejde i firma B\\ Observer alle rækker/kolloner giver 1\\

	Vi opskriver en overgangsmatrice. Det er en stokastisk matrix, hver søjle i P består af
	ikke-negative tal som usmmer til 1. Se også linkmatrixen i Googles page rank
	\[
		P = \begin{pmatrix}
			0.6&0.2&0.4\\
			0.2&0.6&0.4\\
			0.2&0.2&0.2\\
		\end{pmatrix}
	\]
	Den kaldes rægulær, hvis en høj potens $k > 0$ således at $P^k$ kun har strengs positive
	indgange\\ P er regulær da $P^1$ har positive indgange\\
	\begin{theorem}Thm 6.9 (Konvergens)
		Lad $P$ være en rægulær stokastisk matrix. Da gælder:
		\begin{itemize}
			\item $\lambda = 1$ er en egenværdi for P
			\item For en vilkårlig vektor $u_0 \neq 0$ vil $P^ku_0$ konvergere mod en egenvektor
				hørende til egenværdien $\lambda = 1$ en ligevægt.
		\end{itemize}
		dvs.
		\[P^ku_0 \rightarrow w \text{ for } k \rightarrow \infty \text{ hvor } Pw = w\]
	\end{theorem}
	Det giver en måde at beregne en egenvektor for $P$ hørende til lambda=1 uden fx at skulle
	løse ligninger:\\
	Man udregner fx blot $P^ke_1$ for høje potenser K\\
	Hvis P er stor så er det den eneste måde at finde egenvektor på i rimelig tid\\

	Fortsætter vi vores eksempel så har vi 150 dataloger der i år $t_0$ er fordelt:
	\[ u_t = \begin{pmatrix}x_t\\y_t\\z_t\end{pmatrix} \qquad u_1 =
	\begin{pmatrix}60\\50\\40\end{pmatrix}\]

	Vi har \[x_{t+1} = 0.6x_t + 0.2y_t + 0.4z_t \] Og videre, se matrix P

	Vi opskriver udviklingsreglen
	\[u_{t+1} = Pu_t = 
		\begin{pmatrix}
			0.6&0.2&0.4\\
			0.2&0.6&0.4\\
			0.2&0.2&0.2\\
		\end{pmatrix}\begin{pmatrix}x_t\\y_t\\z_t\end{pmatrix}
	\]
	Thm 6.9 forudsiger at arbejdsstyrken på 150 dataloger med tiden vil stabilisere sig mod
	en ligevægt $w$ der kan beregnes som 
	\[ w \approx P^ku_0\] hvor $k$ er stor og $w$ er en egenværdi

	For $k = 5$ har vi et estimat

	\[ = P^5u_0 = 
		\begin{pmatrix}
			0.6&0.2&0.4\\
			0.2&0.6&0.4\\
			0.2&0.2&0.2\\
		\end{pmatrix}^5\begin{pmatrix}60\\50\\40\end{pmatrix}
		\approx 
		\begin{pmatrix}60\\60\\30\end{pmatrix}
	\]

	\section{Principal Component Analysis PCA}
	Vi har et datasæt i eks 3-dimensioner. PCA reducerer til en vektor og en ortogonalvektor\\
	De kan eks. graferes i $\RR^3$. Det vil vi lave Principal Component Analysis. Det vil finde
	de retninger som "datasættet peger mest og mindst i". Det kan simpelt beskrives som
	en generalisering af lineær regression.\\
	Hvorfor? Det tillader fx at reducere datasettets dimensioner på en nogenlunde intelligent
	måde. Eksempelvis 3d underrum til 2d plan\\

\end{document}
